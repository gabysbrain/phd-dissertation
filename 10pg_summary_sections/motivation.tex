%Thesis: Multi-dimensional spaces are better visualized through slice based views

We live in a three-dimensional world.  Ourselves and what we can interact with
are in three dimensions.  We learn about our world by studying the various
phenomena around us.  These phenomena are described as continuous processes.
In the beginning of our  education we study function plots in high school.
These give an intuitive view of one-dimensional phenomena.  By exploring the
relationship between an input factor
and output,
we can build an understanding on the relationship between the two.  We can also
compare one function plot to another. Visual inspection of these plots allows
us to see common patterns. We use our pattern recognition ability to quickly
categorize these different plots into different types of function behavior.
Function plots can also be used to describe two-dimensional phenomena. These
show the effect due to two input factors. In this case we can use the third
dimension or color encoding to show the function value.  From these plots we
can also make general statements about the ``shapes'' of the behavior like how
``peaky'' the function is or if it is monotonically increasing. These shapes
give us intuition into the underlying processes and help us learn about the
world~\cite{Palmer:1999}.

The visual analysis of multiple dimensions is one of the central themes of
visualization research. In principle there are two conceptual types of problems
that amount to two different mental models. (1) Often, the data set is
considered to be truly discrete and projection methods, such as scatterplots
and dimensionality reduction techniques, are used for its analysis.  Typical
examples include business applications, in which one is analyzing customer
data.  The focus of this paper is different in that (2) we are focusing on
continuous multi-dimensional data spaces. For computational purposes, the data
set is then merely a set of points sampled from a continuous phenomenon of
study. This is rather common in simulation and engineering applications or for
the study of continuous algorithmic parameters in modeling environments,
including machine learning applications~\cite{Sedlmair:2014}. Of course, for
such scenarios, projection based visualization might be of help as well.
However, they do not respect the mental model of the object of
study~\cite{Tory:2004}.

Understanding multi-dimensional continuous spaces is difficult. As
three-dimensional beings we have real-world analogs for measurement, angle, and
position in three dimensions. We do not have these once we move beyond three
dimensions though. Nevertheless, visual analysis of these multi-dimensional
spaces has produced insights about the underlying
behavior~\cite{Sedlmair:2014,Gleicher:2016}. The issue is how to show more than
three dimensions on a two-dimensional screen. 

Visualizations of multi-dimensional spaces on a 2D screen must contend with
some sort of reduction of the information. A proper visualization must select
visual encodings that highlight the information we want to see. Any sort of
data reduction requires trade-offs. The best visualization choices acknowledge
any deficiencies to a particular visual encoding. By acknowledging these
deficiencies, we can design tools to compensate for their shortcomings while
still maintaining their advantages. 

Van Wijk and
van Liere introduced the idea of using slice-based views of multi-dimensional
data, called HyperSlice~\cite{Wijk:1993}.
Slicing offers a number of advantages over other multi-dimensional
visualization techniques. Slicing is a direct visualization of the
multi-dimensional object. In contrast to methods like projection or dimension
reduction, slicing does not distort the dimensions in order to display them on
a two-dimensional screen. Since there is no distortion, distances in the visual
representation are directly proportional to distances in the object. 

However, they did not expand on what data types and tasks are involved
in multi-dimensional continuous data analysis.  I investigate
the usefulness of slice-based views of continuous
multi-dimensional datasets. I also identified tasks involved in
multi-dimensional data analysis. The task analysis informed the development of
one- and two-dimensional slice-based views.

\subsection{Multi-dimensional spaces}
\label{sec:motivation:multi-d}

There are a number of domains where one can apply the analysis of continuous
multi-dimensional data.  As of yet, there has not been a comprehensive data and
task analysis for multi-dimensional continuous data analysis. For discrete
data, there are several task 
analyses~\cite{Shneiderman:1996,Brehmer:2013,Amar:2004}. 
However, they are
focused on identifying and selecting particular data items. Continuous datasets
consist of ranges of values as well as functions. Functions can be seen as a
mapping from ranges of numbers to other ranges. The analysis task here is to
study these ranges, their relationships to each other, and the mappings between
them. Tasks addressing these
have not yet been covered by visualization task analyses. Thus,
there is no comprehensive source for what analysis tasks one wants to perform
given a continuous multi-dimensional dataset.  Work in this area has
traditionally focused on developing a specific visualization for a specific
task. For example, topological spines extracts critical points from a scalar
field~\cite{Correa:2011}. One goal of this thesis is to develop this task
taxonomy for visualization of continuous multi-dimensional data.


These domains can be broadly classified into two types based on their analysis
tasks. One type, \emph{manifold} analysis, deals with understanding the
relationship between inputs and outputs. This is a functional relationship.
The user wants to inspect how changes in the inputs (independent variables)
affect the outputs (dependent variables). One can also perform \emph{shape}
analysis. Here, in terms of the analysis tasks, there is no identification of
independent and dependent variables. We look at each of these two in turn.

\subsection{Manifolds}
\label{sec:manifolds}

Studying the mapping between continuous ranges means studying functional
relationships and thus manifolds.  The critical issue is understanding the
relationship between independent and dependent variables.  Subtasks in manifold
analysis include examining critical points, assessing the sensitivity of
parameters, and understanding the shape of the manifold.  
One area where
understanding the manifold is important is analyzing optimization surfaces and
functions.  In this case, the identification of extrema is important for
understanding how many and the relative location of local optima. In addition,
we want to understand the degree to which these are extrema. These can result
in global optimization algorithms ``getting stuck'' in local optima rather than
continuing to search for the global optimum. Optimization algorithms need to be
carefully tuned to properly detect these features and ignore them where
necessary.

Simulations can be used to run experiments that are impractical or impossible
in the real world.  Simulation analysis is another area where the analysis
tasks, in the abstract, are examining functions. If we look back at the weather
simulation from before, the inputs to the function are things like the
temperature and pressure.  The output is, for example, the likelihood of rain
the next day. The function is the simulation itself. Computer simulations are
deterministic.  A deterministic simulation has a fixed mapping from each unique
input parameter configuration to an output value. This is the same as a
functional relationship. The sensitivity and extrema are also important to
simulation analysis.  Thus, these can all be analyzed with visualizations of a
manifold.

To date, manifold visualizations have concentrated on a particular analysis
task or a particular application domain. For example, visualizations of the
Morse-Smale complex~\cite{Gerber:2010} are focused on showing only critical
points of the manifold. As with any visualization designed for a specific task,
they must be used in combination with other views for visual analysis of
domain-specific data. Domain-specific visualizations often used linked views to
show different aspects of data to accomplish multiple tasks at once. However,
they are purpose built for a specific domain. While techniques may transfer
from one domain to another~\cite{Sedlmair:2012}, it is not always clear how.
My goal is to unify these methods to a certain extent.  
Slice-based views of manifolds can be used for a
wide variety of tasks in a wide number of domains.

\subsection{Shapes}
\label{sec:shapes}

One may also want to understand the relationship or correlation between
multiple continuous values. In the manifold analysis case we have the notion
of independent and dependent variables. This classification does not exist 
here.
In this case
we want to study the relationship of all variables. Careful study of the shape
of the dataset can give insight into the relationship between the various
ranges of dimensions of the object. For example, one may want to know if the
overall shape is a sphere, donut, or box. In addition one may be interested in
any kinks or cusps in the dataset. Changes in the gradient and curvature of
the shape are also of great interest. These indicate changes in correlation or
relation.

These two different data types and sets of tasks require different visualization
considerations. Proper visualization for manifold analysis should focus on
the relationships between independent and dependent variables. Visualizations
of shapes do not have this mapping requirement and instead focus on the 
relationships between dimensions. With this categorization in mind, we can now
examine the available visualization techniques to examine these.



Another benefit of the direct visualization is that users do not require
extensive training to understand the visualization. The concept of slicing
through a three-dimensional object is a familiar one. Humans are used to this
even from slicing fruits and vegetables with a knife. This concept of slicing
can be extended from this well-known metaphor to cover multiple slices of
multi-dimensional objects.

Slice views use the horizontal and vertical axes for showing the effects due to
the input parameters. These axes are the most perceptually
uniform~\cite{Stevens:1957} and are considered the most effective
(\autoref{tbl:visual_encodings}). One- or two-dimensional
plots are replicated for each combination of dimensions in order to show more
than two dimensions at once. This promotes familiarity of the visualization.
Once the user has learned to read a single panel, they can apply this knowledge
to the remaining panels. This approach follows the principal of small
multiples~\cite{Archambault:2011}. 

In order to produce a slice plot one needs to first pick a particular
\emph{focus point} in the multi-dimensional space. This focus point determines
which slices are being viewed. Selecting a good focus point a-priori is
difficult. 
It either requires a great deal of luck or careful analysis of the
dataset. This is not always possible. Slice-based views require some sort of
interactive focus point selection. Interactively browsing through the slices
requires interaction controls to give the user control over the focus point.
Furthermore, we need some kind of navigation map to show which focus points the
user has selected so that they do not become lost. Neither of these navigation
aids are well developed at this point. This need for interaction is likely one
of the reasons that slice-based views have not developed as much as projection
or topological techniques. Static views are much easier to include in papers
and don't require explanation prior to use.

The other implementation issue of slice-based views is ensuring that the
visualization can remain interactive. In this case, interactive is defined as
the rate at which the user can maintain their
concentration~\cite{Shneiderman:1987}. This is often defined at 10 frames per
second. It can be difficult to compute a 2D slice of an arbitrary complex
multi-dimensional object. 

