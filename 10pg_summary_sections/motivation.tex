%Thesis: Multi-dimensional spaces are better visualized through slice based views

We live in a three-dimensional world.  
We learn about our world by studying the various
phenomena around us.  These phenomena are described as continuous processes.
In the beginning of our  education, we study function plots in high school.
These give an intuitive view of one-dimensional phenomena.  
By exploring the
relationship between an input factor
and output,
we can build an understanding on the relationship between the two.  We can also
compare one function plot to another. Visual inspection of these plots allows
us to see common patterns. We use our pattern recognition ability to quickly
categorize these different plots into different types of function behavior.
Function plots can also be used to describe two-dimensional phenomena. These
show the effect due to two input factors. In this case we can use the third
dimension or color encoding to show the function value.  From these plots we
can also make general statements about the ``shapes'' of the behavior like how
``peaky'' the function is or if it is monotonically increasing. These shapes
give us intuition into the underlying processes and help us learn about the
world~\cite{Palmer:1999}.

The visual analysis of multiple dimensions is one of the central themes of
visualization research. In principle there are two conceptual types of problems
that amount to two different mental models. Often, the data set is
considered to be truly discrete and projection methods, such as scatterplots
and dimensionality reduction techniques, are used for its analysis.  Typical
examples include business applications, in which one is analyzing customer
data.  The focus here is different in that we are focusing on
continuous multi-dimensional data spaces. For computational purposes, the data
set is then merely a set of points sampled from a continuous phenomenon of
study. This is rather common in simulation and engineering applications or for
the study of continuous algorithmic parameters in modeling environments,
including machine learning applications~\cite{Sedlmair:2014}. Of course, for
such scenarios, projection based visualization might be of help as well.
However, they do not respect the mental model of the object of
study~\cite{Tory:2004}.

Understanding multi-dimensional continuous spaces is difficult. As
three-dimensional beings we have real-world analogs for measurement, angle, and
position in three dimensions. We do not have these once we move beyond three
dimensions though. Nevertheless, visual analysis of these multi-dimensional
spaces has produced insights about the underlying
behavior~\cite{Sedlmair:2014,Gleicher:2016}. The issue is how to show more than
three dimensions on a two-dimensional screen. 
However, any 2D or 3D view of a multi-dimensional space necessarily requires
aggregation of that space.
We can only ``see'' a subsection of the parameter
space at one time.
Therefore, one must create multiple static views, each looking at 
the data from a different perspective.  A scatterplot matrix, for example,
shows a 2D projection of the data for each pair of dimensions.

By allowing for user interaction one is not limited to a
predetermined set of views.  
When the view selection changes then a new view of the data must be built
on the fly.
However, if the visualization system does not respond quickly to 
the user's interaction then the cognitive connection with the visualization
is lost~\cite{Shneiderman:1987} along with the advantage of adding 
interaction in the first place.
Arguments about what exact
response time makes a visualization \emph{interactive} vary.  However,
view updates somewhere between 10fps to 60fps are typically deemed acceptable.

Van Wijk and
van Liere introduced the idea of using slice-based views of multi-dimensional
data, called HyperSlice~\cite{Wijk:1993}.
Slicing offers a number of advantages over other multi-dimensional
visualization techniques. Slicing is a direct visualization of the
multi-dimensional object. In contrast to methods like projection or dimension
reduction, slicing does not distort the dimensions in order to display them on
a two-dimensional screen. Since there is no distortion, distances in the visual
representation are directly proportional to distances in the object. 

There are a number of domains where one can apply the analysis of continuous
multi-dimensional data.  As of yet, there has not been a comprehensive data and
task analysis for multi-dimensional continuous data analysis. For discrete
data, there are several task 
analyses~\cite{Shneiderman:1996,Brehmer:2013,Amar:2004}. 
However, they are
focused on identifying and selecting particular data items. Continuous datasets
consist of ranges of values as well as functions. Functions can be seen as a
mapping from ranges of numbers to other ranges. The analysis task here is to
study these ranges, their relationships to each other, and the mappings between
them. Tasks addressing these
have not yet been covered by visualization task analyses. Thus,
there is no comprehensive source for what analysis tasks one wants to perform
given a continuous multi-dimensional dataset.  Work in this area has
traditionally focused on developing a specific visualization for a specific
task. For example, topological spines extracts critical points from a scalar
field~\cite{Correa:2011}. One goal of this thesis is to develop this task
taxonomy for visualization of continuous multi-dimensional data.

These domains can be broadly classified into two types based on their analysis
tasks. One type, \emph{manifold} analysis, deals with understanding the
relationship between inputs and outputs. This is a functional relationship.
The user wants to inspect how changes in the inputs (independent variables)
affect the outputs (dependent variables). One can also perform \emph{shape}
analysis. Here, in terms of the analysis tasks, there is no identification of
independent and dependent variables. We look at each of these two in turn.

\subsection{Manifolds}
\label{sec:manifolds}

Studying the mapping between continuous ranges means studying functional
relationships and thus manifolds.  The critical issue is understanding the
relationship between independent and dependent variables.  Subtasks in manifold
analysis include examining critical points, assessing the sensitivity of
parameters, and understanding the shape of the manifold.  
One area where
understanding the manifold is important is analyzing optimization surfaces and
functions.  In this case, the identification of extrema is important for
understanding how many and the relative location of local optima. In addition,
we want to understand the degree to which these are extrema. These can result
in global optimization algorithms ``getting stuck'' in local optima rather than
continuing to search for the global optimum. Optimization algorithms need to be
carefully tuned to properly detect these features and ignore them where
necessary.

To date, manifold visualizations have concentrated on a particular analysis
task or a particular application domain. For example, visualizations of the
Morse-Smale complex~\cite{Gerber:2010} are focused on showing only critical
points of the manifold. As with any visualization designed for a specific task,
they must be used in combination with other views for visual analysis of
domain-specific data. Domain-specific visualizations often used linked views to
show different aspects of data to accomplish multiple tasks at once. However,
they are purpose built for a specific domain. While techniques may transfer
from one domain to another~\cite{Sedlmair:2012}, it is not always clear how.
My goal is to unify these methods to a certain extent.  
Slice-based views of manifolds can be used for a
wide variety of tasks in a wide number of domains.

\subsection{Shapes}
\label{sec:shapes}

One may also want to understand the relationship or correlation between
multiple continuous values. In the manifold analysis case we have the notion
of independent and dependent variables. This classification does not exist 
here.
In this case
we want to study the relationship of all variables. Careful study of the shape
of the dataset can give insight into the relationship between the various
ranges of dimensions of the object. For example, one may want to know if the
overall shape is a sphere, donut, or box. In addition one may be interested in
any kinks or cusps in the dataset. Changes in the gradient and curvature of
the shape are also of great interest. These indicate changes in correlation or
relation.

These two different data types and sets of tasks require different visualization
considerations. Proper visualization for manifold analysis should focus on
the relationships between independent and dependent variables. Visualizations
of shapes do not have this mapping requirement and instead focus on the 
relationships between dimensions. With this categorization in mind, we can now
examine the available visualization techniques to examine these.

