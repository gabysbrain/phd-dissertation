
Understanding multi-dimensional spaces is difficult. Visualization can give
us context to help understand the geometry. With the direct visualization
of these multi-dimensional continuous datasets through slice views, we can
use a familiar concept to give context and meaning to a complex task.

Multi-dimensional continuous functions are commonly visualized with 2D slices
or topological views. With Sliceplorer, I explore 1D slices as an alternative
approach to show such functions. My goal with 1D slices is to combine the
benefits of topological views, that is, screen space efficiency, with those of
slices, that is a close resemblance of the underlying function.  I compare 1D
slices to 2D slices and topological views, first, by looking at their
performance with respect to common function analysis tasks. I also demonstrate
3 usage scenarios: the 2D sinc function, neural network regression, and
optimization traces. Based on this evaluation, I characterize the advantages
and drawbacks of each of these approaches, and show how interaction can be used
to overcome some of the shortcomings. 


I also presented Hypersliceplorer, an algorithm for generating 2D
slices of multi-dimensional shapes defined by a simplical mesh.  Often, slices
are generated by using a parametric form and then constraining parameters to
view the slice. In this case, I developed an algorithm to slice a simplical
mesh of any number of dimensions with a two-dimensional slice. In order to get
a global appreciation of the multi-dimensional object, I show multiple slices
by sampling a number of different slicing points and projecting the slices into
a single view per dimension pair. These slices are shown in an interactive
viewer which can switch between a global view (all slices) and a local view
(single slice). I show how this method can be used to study regular polytopes,
differences between spaces of polynomials, and multi-objective optimization
surfaces. 


Finally, I develop a method for predicting the rendering time to display
multi-dimensional data for the analysis of computer simulations using the
HyperSlice~\cite{Wijk:1993} method with Gaussian process model reconstruction.
My method relies on a theoretical understanding of how the data points are
drawn on slices and then fits the formula to a user's machine using practical
experiments.  I also describe the typical characteristics of data when
analyzing deterministic computer simulations as described by the statistics
community.  I then show the advantage of carefully considering how many data
points can be drawn in real time by proposing two approaches of how this
predictive formula can be used in a real-world system.


\subsection{Future}

My work has had a major focus on using direct visualization techniques to
understand multi-dimensional continuous spaces. My intention is that this work
can be expanded upon to herald in a new era of multi-dimensional data analysis.
In my opinion, the major innovations preventing this technique from being used
in a broader application are a library for slicing multi-dimensional spaces and
more user-focused projects.  Building on these two thrusts will move
multi-dimensional continuous data analysis to the mainstream.

One of the reasons for the lack of adoption for slice-based visualization of
multi-dimensional objects is the complete lack of software to generate even
static slice views. There are many libraries for popular data analysis
languages like Python, Javascript, and R. In order to make slice based views
more viable I plan to develop an interactive slice-based visualization software
based on the prototype tools I have already developed. This will lower the cost
of entry of slice-based views of multi-dimensional continuous datasets. The end
result is more users familiar with this visualization type.

In addition, more focused projects with end-users in the form of design
studies~\cite{Sedlmair:2012} will help to develop both the task taxonomy and
the visualization techniques. As part of the task abstraction, we can learn how
these users' tasks fit in with the task and data taxonomy proposed in this
thesis. Then we can refine and extend the task and data taxonomy. This taxonomy
will allow visualization researchers to identify gaps and develop tools to
address them, thus creating more effective visualizations of multi-dimensional
continuous data.


\subsection{Implications}

The main goal of my thesis was to explore what is possible with slice-based
visualizations of continuous multi-dimensional datasets. My hope is that this
work will serve as a basis for an increasing focus on direct visualization of
multi-dimensional objects. Often it seems that the default analysis technique
for more than three dimensions is to reduce
the dimensionality of the
data and then render the reduced data on screen. This suffers from issues of
distortion of distances and relative sizes. The analysis tasks for
multi-dimensional data are all developed around understanding the carefully
chosen dimensions. Hence, transforming these dimensions takes away a lot of
contextual knowledge about the simulation. 

I also hope to bring more attention to continuous multi-dimensional data analysis.
In the visualization community, most of the work on multi-dimensional and high-dimensional
data has focused on the discrete case. There are many task taxonomies, techniques,
and applications for discrete data. My hope with this thesis is that by developing
a task and data taxonomy as well as an in-depth study of direct visualization
techniques will bring similar attention to multi-dimensional continuous data
analysis. There are a number of under-explored application areas in this
field. I have identified some in my own work, but with further research in this
field will bring more knowledge and understanding about how we, as three-dimensional
beings can understand multi-dimensional continuous datasets.





