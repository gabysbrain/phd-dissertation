
Kontinuierliche Daten machen einen gro{\ss}en Teil der physikalischen Daten aus und
zahlreiche Ph{\"a}nomene, die man untersuchen m{\"o}chte werden von unterschiedlichen
Faktoren beeinflusst. Um diese Ph{\"a}nomene zu verstehen, m{\"u}ssen mehrdimensionale
kontinuierliche Daten untersucht werden. Die visuelle Analyse solcher Ph{\"a}nomene
kann viele Erkenntnisse liefern. Es stellt sich jedoch die Frage, wie man etwas
in mehr als drei Dimensionen auf einem 2D-Bildschirm darstellen soll. Ein
Gro{\ss}teil der Analysetools f{\"u}r hochdimensionale Daten (mehr als drei
Dimensionen) konzentriert sich auf diskrete Daten. Diese Methoden k{\"o}nnen jedoch
nicht die gesamte Bandbreite des kontinuierlichen Prozesses darstellen. Die
meisten Methoden zur Visualisierung kontinuierlicher Daten konzentrieren sich
entweder auf ein bestimmtes Gebiet oder eine bestimmte Aufgabe (z.B.\
Optimierung).

Im Rahmen dieser Dissertation suche ich nach M{\"o}glichkeiten, um Universaltools
zur Analyse von mehrdimensionalen kontinuierlichen Daten zu schaffen. Dies tue
ich mittels vier wesentlicher Probleml{\"o}sungsschritte. Erstens f{\"u}hre ich eine
Task Taxonomy f{\"u}r mehrdimensionale kontinuierliche Daten ein. Zweitens
untersuche ich die Verwendung von 1D-Scheiben, um mehrdimensionale
kontinuierliche Funktionen zu verstehen. Drittens entwickelte ich einen
Algorithmus, um 2D-Scheiben von Simplical Meshes zu erzeugen. Somit konnte ich
zeigen, wie diese zum Begreifen von Formen genutzt werden k{\"o}nnen. Viertens habe
ich einen Algorithmus entwickelt, um Scheiben in interaktiver Zeit zu erzeugen.
Dieser Algorithmus macht sich die regelm{\"a}{\ss}ige Geometrie des mehrdimensionalen
Raumes zu Nutze, sowie die GPU-Architektur eines modernen Computers.

Die Ergebnisse dieser Arbeit k{\"o}nnen als Basis f{\"u}r Forschungen an Methoden
direkter Visualisierung von mehrdimensionalen kontinuierlichen Daten genutzt
werden. Mit dieser Arbeit habe ich eine Diskussion {\"u}ber die damit verbundenen
Aufgaben begonnen sowie konkrete Beispiele daf{\"u}r gegeben, wie die
Visualisierungen anhand dieser Aufgaben beurteilt werden k{\"o}nnen. Ich hoffe,
dass die Ergebnisse dieser Arbeit zu weiteren Forschungen an allgemeinen
Methoden zur Visualisierung von mehrdimensionalen kontinuierlichen Daten
f{\"u}hren.

