

Many physical data are continuous and many phenomena that we want to study are
influenced by a number of factors. To understand these phenomena we need to
examine multi-dimensional continuous data.  Visual analysis of these phenomena
can lead to many insights. However, the question remains of how to visualize
something in more than three dimensions on a 2D screen.  Most of the higher
(i.e.\ more than three) dimensional data analysis tools have focused on
discrete data.  These methods cannot represent the full richness of the
continuous process.  Most continuous data visualization methods focus on either
a particular domain area or a particular task (e.g.\ optimization).

In this thesis I explore the possibilities of creating general-purpose tools
for multi-dimensional continuous data analysis. I do this through four key
developments. First, I introduce a task taxonomy for continuous
multi-dimensional data.  Second, I investigated the use of 1D slices to
understand multi-dimensional continuous functions.  Third, I developed an
algorithm to generate 2D slices of simplical meshes and demonstrated how these
can be used to understand shapes.  Forth, I developed an algoritm to render
slices in interactive time. The algorithm takes advantage of regular geometry
of the multi-dimensional space as well as the GPU architecture on a modern
computer.

The results of this work can be used as a basis for research on direct
visualization methods of multi-dimesnsional continuous data. Through this work,
I have started a discussion of the tasks involved and given concrete examples
of how visualizations can be evaluated based on these tasks.  My hope is that
these developments will herald additional research on general methods for 
multi-dimensional continuous data visualization.

