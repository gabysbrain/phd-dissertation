%Thesis: Multi-dimensional spaces are better visualized through slice based views

We live in a three-dimensional world. 
Ourselves and what we can interact with are in three dimensions.
We learn about the world by studying the various phenomena around us.
These phenomena governing the world are described as continuous processes.
In order to learn about the world around us we need to study these processes.
We begin by studying function plots in high 
school. These give an intuitive view of one-dimensional 
phenomena. 
By exploring the relationship
between an input factor
%(temperature, pressure, et cetera) 
and output,
%(the weather) 
we can build an understanding on the relationship between the two.
We can also compare one function plot to another. Visual inspection of these plots
allows us to see common patterns. We can use our pattern recognition ability
to quickly categorize these different plots into different types of function
behavior. These function plots can also be used to describe two-dimensional
phenomena where there are two input factors. In this case we can use the third
dimension or color encoding to show the function value. 
From these plots we can also make general statements about the ``shapes'' of
the behavior like how ``peaky'' the function is or if it is monotonically
increasing. These shapes give us intuition into the underlying processes and
help us learn about the world~\cite{Palmer:1999}.

However, we interact with many phenomena around us that are essentially
multi-dimensional in nature. For example, the weather in a certain location is
determined by the temperature, pressure, humidity, dew point, wind velocity,
and wind direction, among others. A change in any of these factors results in a
change in the weather. Each of these factors can be given a ``spatial
embedding.'' Then, they can be viewed as a dimension of some space.  By
``walking'' or navigating through this space we can observe the effect on the
weather due to changes in these parameters. 

Understanding multi-dimensional continuous spaces is difficult. As
three-dimensional beings we have real-world analogs for measurement,
angle, and position in three dimensions. We do not have these once we
move beyond three dimensions though. Nevertheless, visual analysis of
these multi-dimensional spaces has produced insights about the
underlying behavior~\cite{Sedlmair:2014}. The issue is how to show more
than three dimensions on a two-dimensional screen. 

One strategy is to discretize the dataset through sampling and then use the
wealth of discrete visualization tools available. Much of the work on
multi-dimensional data analysis developed from analysis of abstract data such
as tabular data. These datasets are often recorded from real-world
events such as census, species, or text data and contain many different aspects
about each entity. These datasets are inherently multi- or high-dimensional.
Each different aspect of the data items creates a dimension to be analyzed. For
example, census data consists of the address, age, marital status, among
others. In the case of these data the mental model is that of discrete objects
like humans, plants, or documents. Since the mental model is discrete in this
case it makes sense to use data processing and visualization tools designed for
discrete data.  However, our mental model is a continuous one. Therefore, the
discrete data paradigm breaks our mental model~\cite{Tory:2004a,Liu:2010a}. Rather, we
should use visualization techniques purpose-built for continuous data.

While not as
extensively developed, there is previous work on visualizing multi-dimensional,
continuous data. These techniques can be broadly classified into projection,
topological, and slicing methods.  \ttwnote{put in chart from Jurgen's
presentation (projection vs dim reduction)?} Projection methods attempt to
distort the multi-dimensional object in order to view it on a two-dimensional
screen. With projection techniques we can preserve distance, direction, size,
or angles, but not all of 
these~\cite{Snyder:1987}. 
Depending on the
projection method, we may see radically different representations.  The issue
is that it is not clear from the resulting visualization what sort of
transformation was performed on the data.  Thus it can be difficult to
reconstruct the mental model of the multi-dimensional object.  One of the most
often seen multi-dimensional projection techniques is the Schlegel
diagram~\cite{schlegel} which picks a ``face'' of a polytope and projects the
remaining faces inside it. Thus this technique only works for 4D polytopes.
Topological methods search the continuous dataset for values of interest, such
as critical points or contours.  Topological visualization techniques also
suffer from the issue of unclear transformation. It is difficult to relate the
resulting visualization back to features in the multi-dimensional object.

Slice-based views of multi-dimensional continuous spaces have not been explored
as extensively as other options.  This work began with the advent of
HyperSlice~\cite{Wijk:1993}. HyperSlice provides the framework for visualizing
multi-dimensional continuous objects as a set of two-dimensional slices
(\autoref{fig:slicing_overview}). HyperSlice extends the idea of slicing
from medical imaging to any number of dimensions. There are $d \choose 2$
subpanels, one for each pair of dimensions. Each panel shows a 2D slice
of the object. The horizontal axis shows one dimension and the vertical 
axis shows another dimension. With 2D slices of solid multi-dimensional objects
color is often used to encode value. 

My work is inspired by the HyperSlice technique. Van Wijk and van Liere
introduced the idea of using slice-based views of multi-dimensional data.
However, they did not expand on what data types and tasks are involved in
multi-dimensional continuous data analysis.  I build on their work,
investigating the usefulness of slice-based views of continuous
multi-dimensional datasets. I also identified tasks involved in
multi-dimensional data analysis. These tasks informed the development of one-
and two-dimensional slice-based views.

In this thesis I will explore
the possibilities of these slice-based views. Through a number of case
study examples, I will demonstrate the power of these views and ways to
address their shortcomings.

\begin{figure}
  \centering
  %\includegraphics{slicing_overview}
  \label{fig:slicing_overview}
  \caption{
    \ttwnote[nomarginclue]{diagram of slicing}
  }
\end{figure}

%\tmnote{
%%- I think your reasoning though is not so much as looking at wheather. Rather,
%%I would argue that we have learned about the world through looking at 1D plot
%%in school or by comparing (and equating) things into classes of similar things
%%(hence, shape is a nice thing for us to reason about.
%- then you can talk about extending reasoning of 3D world to higher
%dimensions, that it is hard and all.
%- then I would argue about continuous vs. discrete. to me it is important to
%understand the mental model of how we reason about a phenomenon. somtimes
%discrete is important (some examples) sometimes continues (simulations,
%incluence of a parameter)
%- then you can reason about what Vis techniques we have for understanding
%cont. spaces
%}


\section{Multi-dimensional spaces}
\label{sec:motivation:multi-d}

There are a number of domains where one can apply the analysis of continuous
multi-dimensional data.  As of yet, there has not been a comprehensive data and
task analysis for multi-dimensional continuous data analysis. For discrete
data, there are several task 
analyses~\cite{Shneiderman:1996,Brehmer:2013,Amar:2004}. 
However, they are
focused on identifying and selecting particular data items. Continuous datasets
consist of ranges of values as well as functions. Functions can be seen as a
mapping from ranges of numbers to other ranges. The analysis task here is to
study these ranges.  Tasks addressing studying the mappings or studying the
relationship between ranges have not yet been covered by visualization task
analyses. Thus, there is no comprehensive source for what analysis tasks one
wants to perform given a continuous multi-dimensional dataset.  Work in this
area has traditionally focused on developing a specific visualization for a
specific task. For example, topological spines extracts critical points from a
scalar field~\cite{Correa:2011}. One goal of this thesis is to develop this
task taxonomy for visualization of continuous multi-dimensional data.

%Often work on continuous multi-dimensional data analysis is done in the context
%of design studies~\cite{Sedlmair:2012}. The issue is transferability, while
%important, is not always considered. 

%Each of these application scenarios are often treated individually. These all
%fit under a unified visualizaiton method though.

These domains can be broadly classified into two types based on their analysis
tasks. One type, \emph{manifold} analysis, deals with understanding the
relationship between inputs and outputs. This is a functional relationship.
The user wants to inspect how changes in the inputs (independent variables)
affect the outputs (dependent variables). One can also perform \emph{shape}
analysis. Here, in terms of the analysis tasks, there is no identification of
independent and depdendent variables. We look at each of these in turn.

\subsection{Manifolds}
\label{sec:manifolds}

Studying the mapping between continuous ranges means studying functional
relationships and thus manifolds.  The critical issue is understanding the
relationship between independent and dependent variables.  Subtasks in manifold
analysis include examining critical points, assessing the sensitivity of
parameters, and understanding the shape of the manifold.  Analyzing functions
is of course within the realm of manifold analysis.  Another area where
understanding the manifold is important is analyzing optimization surfaces and
functions.  In this case, the identification of extrema is important for
understanding how many and the relative location of local optima. In addition,
we want to understand the degree to which these are extrema. These can result
in global optimization algorithms ``getting stuck'' in local optima rather than
continuing to search for the global optimum. Optimization algorithms need to be
carefully tuned to properly detect these features and ignore them where
necessary.

Simulations can be used to run experiments that are impractical or impossible
in the real world.  Simulation analysis is another area where the analysis
tasks, in the abstract, is examining functions. If we look back at the weather
simulation from before, the inputs to the function are things like the
temperature and pressure.  The output is, for example, the likelihood of rain
the next day. The function is the simulation itself. Computer simulations are
deterministic.  A deterministic simulation has a mapping from each unique input
parameter configuration to an output value. This is the same as a functional
relationship. The sensitivity and extrema are also important to simulation
analysis.  Thus, these can all be analyzed with visualizations of a manifold.

%Some computer simulations are considered stochastic. That is, their output
%is the determined by a random number generator as well as input values.
%Stochastic simulations can be converted to deterministic by treating the 
%random number generator seed as another parameter.

To date, manifold visualizations have concentrated on a particular analysis
task or a particular application domain. For example, visualizations of the
Morse-Smale complex~\cite{MSvis} are focused on showing only critical points of
the manifold. As with any visualization designed for a specific task, they must
be used in combination with other views for visual analysis of domain-specific
data. Domain-specific visualizations often used linked views to show different
aspects of data to accomplish multiple tasks at once. However, they are purpose
built for a specific domain. While techniques may transfer from one domain to
another~\cite{Seldmair:2012}, it is not always clear how.
My goal is to unify these methods to a certain extent.
As I will show in \autoref{chp:sliceplorer}, slice-based views of manifolds
can be used for a wide variety of tasks in a wide number of domains.

\subsection{Shapes}
\label{sec:shapes}

One may also want to understand the relationship or correlation between
multiple continuous values. This is in contrast to studying the relationship
between independent and dependent variables in the manifold case. In this case
we want to study the relationship of all variables. Careful study of the shape
of the dataset can give insight into the relationship between the various
ranges of dimensions of the object. For example, one may want to know if the
overall shape is a sphere, donut, or box. In addition one may be interested in
any kinks or cusps in the dataset. \ttwnote{why?} These do not necessarily need
to be true cusps. Changes in the gradient and curvature of the shape are also
of great interest.

These analysis tasks of multi-dimensional shapes can be applied to a number of
different areas. The study of polytopes is one such area and perhaps the most
direct application of understanding multi-dimensional continuous shapes.
Polytopes are the multi-dimensional generalization of polyhedra and polygons.
The tasks are to understand the symmetries and patterns making up the
polytopes~\cite{Ziegler:2012}. Perhaps a less obvious connection is the
analysis of the tradeoff curves in multi-objective optimization. Since we are
performing optimization we are interested in the tradeoffs amongst all the
non-dominated points~\cite{Kung:1975}. This is also known as the Pareto
front. In this case, the user wants to understand what are the costs of
reducing one or more parameters in order to increase the value of others. Cusps
or large changes in curvature in these datasets are important since they show
fundamental changes in the rate of tradeoff.  With a proper view of a
multi-dimensional object we can also view differences betweeen two objects
directly. With 1-, 2-, or 3-D objects we show these differences directly and
can understand them.  Showing a difference object requires the user to reconstruct
the two objects at once. This requires quite a lot of mental processing.
However, the two-stage processing of showing a difference object and then
performing dimension reduction makes it difficult to view difference objects
with dimension reduction. Slicing, as a direct visualization technique, does
not have this problem. \ttwnote{confusing, use images/diagrams to explain?}

These two different data types and sets of tasks require different visualization
considerations. Proper visualization for manifold analysis should focus on
the relationships between independent and dependent variables. Visualizations
of shapes do not have this mapping requirement and instead focus on the 
relationships between dimensions. With this categorization in mind, we can now
examine the available visualization techniques to examine these.


%
\section{Mental models}
\label{sec:mental_models}

\ttwwarning{write}

Mental models~\cite{Liu:2010a}
\cite{McNeil:2015}

\begin{itemize}
\item what about discrete data?
\item why is this important?
\item issue: ranges - why? where do they come from?
\item issue: treating dimensions equally
\item issue: understanding mapping relationships
\end{itemize}



\section{Visualizing multi-dimensional continuous spaces}
\label{sec:multi-d-challenges}

Understanding multi-dimensional space is difficult. As humans, we simply do not
have the spatial analogs in more than three dimensions. A number of methods
have been developed to extract specific features from the multi-dimensional
object. For example, when studying polytopes, the number of faces and
symmetries is very important~\cite{Ziegler:2012}. However, these only produce
an answer without sufficient context. They do not necessarily give any
intuition as to how to transfer our three-dimensional knowledge to
multi-dimensional spaces.

%\ttwnote{how visualization can help}
%\ttwnote{some sort of insight generation from context?}

Visualizations of multi-dimensional spaces on a 2D screen must contend with
some sort of reduction of the information. A proper visualization must select
visual encodings that highlight the information we want to see. Any sort of
data reduction requires trade-offs. The best visualization choices acknowledge
any difficiencies to a particular visual encoding. By acknowledging these
difficiencies, we can design tools to compensate for their shortcomings while
still maintaining their advantages. Therefore, it is worth first looking at the
possible mappings of data to visual elements. Then, I present commonly used
visual encodings of multi-dimensional continuous data using these mappings.

\subsection{Encoding multi-dimensional data}

Multi-dimensional continuous data consists of a set of continuous ranges, one
for each dimension. In the case of manifold analysis, each of these ranges can
be additionally classified as ``dependent'' or ``independent'' depending on
which side of the mapping they are on. Typical visualization practice is to
give each dimension a separate visual channel. There are a number of possible
visual channels that have been identified.  The ranking of effectiveness of
visual channels (shown in \autoref{tbl:visual_encodings}) was proposed by
Bertin~\cite{Bertin:1967} and confirmed through experiments by Cleveland and
McGill~\cite{Cleveland:1984}, Mackinlay~\cite{Mackinlay:1986}, and Heer and
Bostock~\cite{Heer:2010}.  Munzner~\cite{Munzner:2014} provides a summary of
the results. We are also limited in how many channels we can use
simulataneously. According to Ware~\cite{Ware:2004}, certain channels, such as
red and green are not visually seperable. 

\begin{table}
  \caption{Rankings of visual encodings of quantiative data}
  \label{tbl:visual_encodings}
  \begin{tabular}{llll}
    \cite{Bertin:1967} & \cite{Cleveland:1984} & \cite{Mackinlay:1986} & \cite{Munzner:2014} \\
     Position & Position along a common scale & Position & Position on common scale \\
     Size & Position along identical, nonaligned scales & Length & Position on unaligned scale \\
     (Grey) Value & Length & Angle & Length (1D size) \\
     Texture & Direction & Slope & Tilt/angle \\
     Color & Angle & Area & Area (2D size) \\
     Orientation & Area & Volume & Depth (3D position) \\
     Shape & Volume & Density & Color luminance\\
     & Curvature & Color saturation & Color saturation \\
     & Densities & Color hue & Curvature \\
     & Shading & & Volume (3D size) \\
     & Color saturation &           & 
  \end{tabular}
\end{table}

The difficulty of visualizing a continuous multi-dimensional space on a
two-dimensional screen brings a number of challenges. We treat each dimension
separately, thus, we need several different visual channels. However, there are
simply not enough visual channels available to draw a 15-dimensional object in
a single view. This is further complicated by the fact that separate visual
channels are not necessarilly visually seperable.  Furthermore, each dimension
of the multi-dimensional object under study is treated equally. For example, no
particular axis of a polytope is more important than any other.  We should
encode each dimension using equally weighted effectiveness channels.  With
fewer channels available than data dimensions we either need to reduce the data
or use multiple views to properly visualize the data.


\subsection{Methods}

The common taxonomy of how to view multi-dimensional data on screen is based on
discrete data analysis. There, there are two categories: dimension reduction or
projection. With continuous data, though there is a third possibility, that of
slicing. Therefore, I view the taxonomy of \emph{continuous} multi-dimensional
data analysis methods into two categories. Data-driven methods include both
projection and dimension reduction and reduce the dimensionality of the data
before visualization. View-based methods reduce the data during the
visualization. Slicing is a view-based method.

Purely data-driven methods are commonly known as feature selection or dimension
reduction. The goal is to find a subset of dimensions that are critical to
understanding the dataset. Topological techniques take this a step further.
They discard all spatial information about the dataset and only concentrate on
the difference in function value, as in the Morse-Smale
complex~\cite{Gyulassy:2012a}, or evolution of contours, as in the contour
tree~\cite{Carr:2003a}.  Projections also synthesize the dataset into new
dimensions to show using visual channels. Principal component
analysis~\cite{Holbrey:2006} is a popular choice in this area. This rotates the
space and thus produces new set of dimensions that are a linear combination of
the input dimensions. Even this relatively simple operation (a rotation) can be
difficult to understand. For example, iPCA~\cite{Jeong:2009a} was a tool to
help users understand the effects of the dimensional transformation.

View-based methods try and produce multiple linked views of a multi-dimensional
dataset from different angles. Each view shows a subset of the dimensions.
This way we can use a proper set of visual channels for each view.  We use
interaction to link these different views. These multiple, coordinated, linked
views have been one of the biggest success stories from the visualization
community~\cite{Rao:1994}. The traditional HyperSlice~\cite{Wijk:1993}
technique falls into this category. Each panel of the HyperSlice view shows two
of the input dimensions and the value is encoded with color.
The views are linked through the focus point selection. Changing the focus point
in one sub-plot updates the other sub-plots.

My work focuses on the exploration of the combination of data-driven and
view-based methods.  Data-driven methods reduce the data in a way that we can
get a global overview of the dataset. View-based methods are much more detailed
but can only produce a local view of the data. By combining these methods I can
achieve both a global overview as well as an on-demand local view of the
dataset in a single visualization. 



\section{Slices}
\label{sec:slicing-advantages}

Slicing offers a number of advantages over other multi-dimensional
visualization techniques. Slicing is a direct visualization of the
multi-dimensional object. In contrast to methods like projection or dimension
reduction, slicing does not distort the dimensions in order to display them on
a two-dimensional screen. Since there is no distortion, distances in the visual
representation are directly proportional to distances in the object. This is
one of the reasons that slicing is popular in the medical imaging community.
Sizes of organs or tumors can be measured visually on screen. Additionally,
relative sizes correspond to what a doctor would expect to see in the body.
Multi-dimensional data is abstract. It lacks the correspondence to real-world
objects that the medical community uses to understand their two- and
three-dimensional datasets. Nevertheless, it is still important in
multi-dimensional data analysis to properly estimate distances and relative
sizes. 
\ttwnote{example}

Another benefit of the direct visualization is that users do not require
extensive training to understand the visualization. The concept of slicing
through a three-dimensional object is a familiar one. Humans are used to this
even from cooking. This concept of slicing can be extended from this 
well-known metaphor to cover multiple slices of multi-dimensional objects.

Slice views use the horizontal and vertical axes for showing the effects due to
the input parameters. These axes are the most perceptually
uniform~\cite{Stevens:1957} and are considered the most effective
(\autoref{tbl:enc_effectiveness}). One- or two-dimensional
plots are replicated for each combination of dimensions in order to show more
than two dimensions at once. This promotes familiarity of the visualization.
Once the user has learned to read a single panel, they can apply this knowledge
to the remaining panels. This approach follows the principal of small
multiples~\cite{Archambault:2011}. \ttwnote{maybe move this to a section on
general vis techniques}

In order to produce a slice plot one needs to first pick a particular
\emph{focus point} in the multi-dimensional space. This focus point determines
which slices are being viewed. Selecting a good focus point a-priori is
difficult. 
\ttwnote{from michael: this is a challenge again. imho it might be clearer to list all challenges in the same spot. I always liked the spectrum between toplogical approaches and Hyperslices. Then you can say that your goal is to combine their benefits and reduce their weaknesses and how you address these challenges.}
It either requires a great deal of luck or careful analysis of the
dataset. This is not always possible. Slice-based views require some sort of
interactive focus point selection. Interactively browsing through the slices
requires interaction controls to give the user control over the focus point.
Furthermore, we need some kind of navigation map to show which focus points the
user has selected so that they do not become lost. Neither of these navigation
aids are well developed at this point. This need for interaction is likely one
of the reasons that slice-based views have not developed as much as projection
or topological techniques. Static views are much easier to include in papers
and don't require explanation prior to use.

The other implementation issue of slice-based views is ensuring that the
visualization can remain interactive. In this case, interactive is defined as
the rate at which the user can maintain their
concentration~\cite{Shneiderman:1987}. This is often defined at 10 frames per
second. It can be difficult to compute a 2D slice of an arbitrary complex
multi-dimensional object. 






\section{Upcoming}
\label{sec:thesis_outline}

\ttwnote{expand}
In my own work, I address these two critical issues. The navigation issue can
be solved with sampling. With both Sliceplorer (\autoref{chp:sliceplorer}) and
HyperSliceplorer (\autoref{chp:hypersliceplorer}), I sample over a number of
focus points and display them all at once. In \autoref{chp:rendering}, I
discuss how we can take advantage of the multi-dimensional geometry to allow
interactive-speed browsing of the focus points. My goal is to highlight the
advantages of slice-based methods for multi-dimensional data analysis. At the
same time I want to address the limitations. The end goal is to bring
slice-based views into the standard toolbox of visualizations.



