%Thesis: Multi-dimensional spaces are better visualized through slice based views

We live in a three-dimensional world. 
Ourselves and what we can interact with are in three dimensions.
\ttwnote{bridge sentence}
The phenomena governing the world are described as continuous processes.
In order to learn about the world around us we need to study these processes.
We begin by studying function plots in high 
school\ttwnote{figure?}. These give an intuitive view of one-dimensional 
phenomena. 
By exploring the relationship
between an input factor
%(temperature, pressure, et cetera) 
and output,
%(the weather) 
we can build an understanding on the relationship between the two.
We can also compare one function plot to another. Visual inspection of these plots
allows us to see common patterns. We can use our pattern recognition ability
to quickly categorize these different plots into different types of function
behavior. These function plots can also be used to describe two-dimensional
phenomena where there are two input factors. In this case we can use the third
dimension or color encoding to show the function value \ttwnote{figure?}. 
From these plots we can also make general statements about the ``shapes'' of
the behavior like how ``peaky'' the function is or if it is monotonically
increasing. These shapes give us intuition into the underlying processes and
help us learn about the world\ttwnote{can I find a pattern recognition reference?}\ttwnote{or maybe a reference from gestalt principles about grouping shapes together?}.

However, we interact with many phenomena around us that are essentially
multi-dimensional in nature. For example, the weather in a certain location is
determined by the temperature, pressure, humidity, dew point, wind velocity,
and wind direction, among others. A change in any of these factors results in a
change in the weather. Each of these factors can be given a ``spatial
embedding.'' Then, they can be viewed as a dimension of some space.  By
``walking'' or navigating through this space we can observe the effect on the
weather due to changes in these parameters. 

Understanding multi-dimensional continuous spaces is difficult. As
three-dimensional beings we have real-world analogs for measurement,
angle, and position in three dimensions. We do not have these once we
move beyond three dimensions though. Nevertheless, visual analysis of
these multi-dimensional spaces has produced insights about the
underlying behavior~\cite{Sedlmair:2014}. The issue is how to show more
than three dimensions on a two-dimensional screen. 

One strategy is to discretize the dataset through sampling and then use the
wealth of discrete visualization tools available. Much of the work on
multi-dimensional data analysis developed from analysis of abstract data such
as tabular data\ttwnote{ref}. These datasets are often recorded from real-world
events such as census, species, or text data and contain many different aspects
about each entity. These datasets are inherently multi- or high-dimensional.
Each different aspect of the data items creates a dimension to be analyzed. For
example, census data consists of the address, age, marital status, among
others. In the case of these data the mental model is that of discrete objects
like humans, plants, or documents. Since the mental model is discrete in this
case it makes sense to use data processing and visualization tools designed for
discrete data.  However, our mental model is a continuous one. Therefore, the
discrete data paradigm breaks our mental model~\cite{Tory:2004a}. Rather, we
should use visualization techniques purpose-built for continuous data.

\ttwnote{shorten this and move details to specific section} While not as
extensively developed, there is previous work on visualizing multi-dimensional,
continuous data. These techniques can be broadly classified into projection,
topological, and slicing methods.  \ttwnote{put in chart from Jurgen's
presentation (projection vs dim reduction)?} Projection methods attempt to
distort the multi-dimensional object in order to view it on a two-dimensional
screen. With projection techniques we can preserve distance, direction, size,
or angles, but not all of 
these~\cite{Snyder:1987}. 
Depending on the
projection method, we may see radically different representations.  The issue
is that it is not clear from the resulting visualization what sort of
transformation was performed on the data.  Thus it can be difficult to
reconstruct the mental model of the multi-dimensional object.  One of the most
often seen multi-dimensional projection techniques is the Schlegel
diagram~\cite{schlegel} which picks a ``face'' of a polytope and projects the
remaining faces inside it. Thus this technique only works for 4D polytopes.
Topological methods search the continuous dataset for values of interest, such
as critical points or contours.  Topological visualization techniques also
suffer from the issue of unclear transformation. It is difficult to relate the
resulting visualization back to features in the multi-dimensional object.

Slice-based views of multi-dimensional continuous spaces have not been explored
as extensively as other options.  This work began with the advent of
HyperSlice~\cite{Wijk:1993}. HyperSlice provides the framework for visualizing
multi-dimensional continuous objects as a set of two-dimensional slices
(\autoref{fig:slicing_overview}). HyperSlice extends the idea of slicing
from medical imaging to any number of dimensions. There are $d \choose 2$
subpanels, one for each pair of dimensions. Each panel shows a 2D slice
of the object. The horizontal axis shows one dimension and the vertical 
axis shows another dimension. With 2D slices of solid multi-dimensional objects
color is often used to encode value. 

My work is inspired by the HyperSlice technique. Van Wijk and van Liere
introduced the idea of using slice-based views of multi-dimensional data.
However, they did not expand on what data types and tasks are involved in
multi-dimensional continuous data analysis.  I build on their work,
investigating the usefulness of slice-based views of continuous
multi-dimensional datasets. I also identified tasks involved in
multi-dimensional data analysis. These tasks informed the development of one-
and two-dimensional slice-based views.

In this thesis I will explore
the possibilities of these slice-based views. Through a number of case
study examples, I will demonstrate the power of these views and ways to
address their shortcomings.

\begin{figure}
  \centering
  %\includegraphics{slicing_overview}
  \label{fig:slicing_overview}
  \caption{
    \ttwnote[nomarginclue]{diagram of slicing}
  }
\end{figure}

%\tmnote{
%%- I think your reasoning though is not so much as looking at wheather. Rather,
%%I would argue that we have learned about the world through looking at 1D plot
%%in school or by comparing (and equating) things into classes of similar things
%%(hence, shape is a nice thing for us to reason about.
%- then you can talk about extending reasoning of 3D world to higher
%dimensions, that it is hard and all.
%- then I would argue about continuous vs. discrete. to me it is important to
%understand the mental model of how we reason about a phenomenon. somtimes
%discrete is important (some examples) sometimes continues (simulations,
%incluence of a parameter)
%- then you can reason about what Vis techniques we have for understanding
%cont. spaces
%}


\section{Multi-dimensional spaces}
\label{sec:motivation:multi-d}

There are a number of varied domains where one can apply the analysis of
continuous multi-dimensional data.
\ttwnote{contribution: taxonomy of domains}

%Often work on continuous multi-dimensional data analysis is done in the context
%of design studies~\cite{Sedlmair:2012}. The issue is transferability, while
%important, is not always considered. 

%Each of these application scenarios are often treated individually. These all
%fit under a unified visualizaiton method though.

These domains can be broadly classified into two types based on their analysis
goals. One type, \emph{manifold} analysis, deals with understanding the
relationship between inputs and outputs. This is a functional relationship.
The user wants to inspect how changes in the inputs (independent variables)
affect the outputs (dependent variables). One can also perform \emph{shape}
analysis. Here, in terms of the analysis goals, there is no identification of
independent and depdendent variables. \ttwnote{expand this a bit}

\subsection{Manifolds}
\label{sec:manifolds}

Studying manifolds means studying functional relationships. The critical issue
is understanding the relationship between independent and dependent factors.
\ttwnote{subtasks}. Analyzing functions is of course within the realm of
manifold analysis. \ttwnote{examples of work} Another area where understanding
the manifold is important is analyzing optimization surfaces and functions.
\ttwnote{want to see number and relative depth of bowls/peaks same as sens
analysis} \ttwnote{algo can get stuck}

\ttwnote{example} Simulation analysis is another area where the analysis tasks,
in the abstract, is examining functions. \ttwnote{latex equation here?} If we
look back at the weather simulation from before, the inputs to the function are
things like the temperature and pressure. The output is, for example, the
likelihood of rain the next day. The function is the simulation itself.
Computer simulations are deterministic. Stochastic simulations can be converted
to deterministic by treating the random number generator seed as another
parameter.  A deterministic simulation has a mapping from each unique input
parameter configuration to an output value. This is the same as a functional
relationship. The sensitivity, extrema, and \ttwnote{something} are also
important to simulation analysis. \ttwnote{what is the goal?}
Thus, these can all be analyzed with similar visualizations.

\subsection{Shapes}
\label{sec:shapes}

One may also want to understand the relationship or correlation between
multiple continuous values. This is in contrast to studying the relationship
between independent and dependent variables in the manifold case. In this case
we want to study the relationship of all variables. Careful study of the shape
of the dataset can give insight into the overall shape of the object. For
example, one may want to know if the multi-dimensional object is shaped like a
sphere, donut, or box.
In addition one may be interested in any kinks or cusps in the dataset.
\ttwnote{why?} These do not necessarily need to be true cusps. Changes in
the gradient and curvature of the shape are also of great interest.
%Detecting these changes can be difficult to do numerically so 

These analysis tasks of multi-dimensional shapes can be applied to a number of
different areas. The study of polytopes is one such area and perhaps the most
direct application of understanding multi-dimensional continuous shapes.
Polytopes are the multi-dimensional generalization of polyhedra and polygons.
The tasks are to understand the symmetries and patterns making up the
polytopes~\cite{Ziegler:2012}. Perhaps a less obvious connection is the
analysis of the tradeoff curves in multi-objective optimization. Since we are
performing optimization we are interested in the tradeoffs amongst all the
non-dominated points~\cite{nondominated}. This is also known as the Pareto
front. In this case, the user wants to understand what are the costs of
reducing one or more parameters in order to increase the value of others. Cusps
or large changes in curvature in these datasets are important since they show
fundamental changes in the rate of tradeoff.  With a proper view of a
multi-dimensional object we can also view differences betweeen two objects
directly. With 1-, 2-, or 3-D objects we show these differences directly and
can understand them.  \ttwnote{showing a difference object is hard enough}
However, the two-stage processing of showing a difference object and then
performing dimension reduction makes it difficult to view difference objects
with dimension reduction. Slicing, as a direct visualization technique, does
not have this problem.

\ttwnote{conclude, move on}



\section{Challenges with multi-dimensional spaces}
\label{sec:multi-d-challenges}

\ttnote{this section should be about why multi-D is hard}

\ttwnote{too unorganized... maybe turn it into a technique overview section}

Given the importantance of multi-dimensional data analysis, a number of
projects from the visualization community, as well as others, have worked on
this problem. 
\ttwnote{rewrite this paragraph}
\ttwnote{introduce each section of the motivation with what's to come} 

Understanding multi-dimensional space is difficult. As humans, we simply do not
have the spatial analogs in more than three dimensions. A number of methods
have been developed to extract specific features from the multi-dimensional
object. For example, when studying polytopes, the number of faces and
symmetries is very important~\cite{Ziegler:2012}. However, these only produce
an answer without sufficient context. They do not necessarily give any
intuition as to how to transfer our three-dimensional knowledge to
multi-dimensional spaces.

\ttwnote{how visualization can help}
\ttwnote{some sort of insight generation from context?}

The difficulty of visualizing a continuous multi-dimensional space on a
two-dimensional screen brings a number of challenges. We treat each dimension
separately, thus, we need several different visual channels. The ranking of
effectiveness of visual channels (shown in \autoref{tbl:visual_encodings}) was
proposed by Bertin~\cite{Bertin:1967} and confirmed through experiments by
Cleveland and McGill~\cite{Cleveland:1984}, Mackinlay~\cite{Mackinlay:1986},
and Heer and Bostock~\cite{Heer:2010}.  Munzner~\cite{Munzner:2014} provides a
summary of the results. In addition, there are many others. We are also limited
in how many channels we can use simulataneously. According to
Ware~\cite{Ware:2004}, many channels, such as red and green are not visually
seperable. Furthermore, each dimension of the multi-dimensional object under
study is often treated equally. For example, no particular axis of a polytope
is more important than any other.  We should encode each dimension using
equally weighted effectiveness channels.  With fewer channels available than
data dimensions we either need to reduce the data or use multiple views to
properly visualize the data.

\begin{table}
  \caption{Rankings of visual encodings of quantiative data}
  \label{tbl:visual_encodings}
  \begin{tabular}{llll}
    \cite{Bertin:1967} & \cite{Cleveland:1984} & \cite{Mackinlay:1986} & \cite{Munzner:2014} \\
     Position & Position along a common scale & Position & Position on common scale \\
     Size & Position along identical, nonaligned scales & Length & Position on unaligned scale \\
     (Grey) Value & Length & Angle & Length (1D size) \\
     Texture & Direction & Slope & Tilt/angle \\
     Color & Angle & Area & Area (2D size) \\
     Orientation & Area & Volume & Depth (3D position) \\
     Shape & Volume & Density & Color luminance\\
     & Curvature & Color saturation & Color saturation \\
     & Densities & Color hue & Curvature \\
     & Shading & & Volume (3D size) \\
     & Color saturation &           & 
  \end{tabular}
\end{table}

Purely data-driven methods are commonly known as feature selection or dimension
reduction. The goal is to find a subset of dimensions that are critical to
understanding the dataset. Topological techniques take this a step further.
They discard all spatial information about the dataset and only concentrate on
the difference in function value, as in the Morse-Smale
complex~\cite{Gyulassy:2012a}, or evolution of contours, as in the contour
tree~\cite{Carr:2003a}.  Projections also synthesize the dataset into new
dimensions to show using visual channels. Principal component
analysis~\cite{PCA} is a popular choice in this area. This rotates the space
and thus produces new set of dimensions that are a linear combination of the
input dimensions. Even this relatively simple operation (a rotation) can be
difficult to understand. For example, iPCA~\cite{Jeong:2009a} was a tool to
help users understand the effects of the dimensional transformation.

View-based methods try and produce multiple linked views of a multi-dimensional
dataset from different angles. Each view shows a subset of the dimensions.
This way we can use a proper set of visual channels for each view.  We use
interaction to link these different views. These multiple, coordinated, linked
views have been one of the biggest success stories from the visualization
community~\cite{Rao:1994}. The traditional HyperSlice~\cite{Wijk:1993}
technique falls into this category. Each panel of the HyperSlice view shows two
of the input dimensions and the value is encoded with color.

\ttwnote{conclude and transition}


%
\section{Mental models}
\label{sec:mental_models}

\ttwwarning{write}

Mental models~\cite{Liu:2010a}
\cite{McNeil:2015}

\begin{itemize}
\item what about discrete data?
\item why is this important?
\item issue: ranges - why? where do they come from?
\item issue: treating dimensions equally
\item issue: understanding mapping relationships
\end{itemize}


\section{advantages of slicing}
\label{sec:slicing-advantages}

\ttwnote{most important aspects of analysis on most effective axes}

\begin{itemize}
\tightlist
\item
  stuff from Ware
\item
  examples
\item
  projection vs dim reduction vs slices
\item
  tasks?
\end{itemize}

\ttwnote{also talk about shapes vs manifolds}



