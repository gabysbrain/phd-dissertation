
\section{Future}

My work has had a major focus on using direct visualization techniques to
understand multi-dimensional continuous spaces. My intention is that this work
can be expanded upon to herald in a new era of multi-dimensional data analysis.
In my opinion, the major innovations preventing this technique from being used
in a broader application are a library for slicing multi-dimensional spaces and
more user-focused projects.  Building on these two thrusts will move
multi-dimensional continuous data analysis to the mainstream.

One of the reasons for the lack of adoption for slice-based visualization of
multi-dimensional objects is the complete lack of software to generate even
static slice views. There are many libraries for popular data analysis
languages like Python, Javascript, and R. In order to make slice based views
more viable I plan to develop an interactive slice-based visualization software
based on the prototype tools I have already developed. This will lower the cost
of entry of slice-based views of multi-dimensional continuous datasets. The end
result is more users familiar with this visualization type.

In addition, more focused projects with end-users in the form of design
studies~\cite{Sedlmair:2012} will help to develop both the task taxonomy and
the visualization techniques. As part of the task abstraction, we can learn how
these users' tasks fit in with the task and data taxonomy proposed in this
thesis. Then we can refine and extend the task and data taxonomy. This taxonomy
will allow visualization researchers to identify gaps and develop tools to
address them, thus creating more effective visualizations of multi-dimensional
continuous data.

