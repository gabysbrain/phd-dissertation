
% No section thingee

Multi-dimensional continuous functions are commonly visualized with 2D slices
or topological views. With Sliceplorer, I explore 1D slices as an alternative
approach to show such functions. My goal with 1D slices is to combine the
benefits of topological views, that is, screen space efficiency, with those of
slices, that is a close resemblance of the underlying function.  I compare 1D
slices to 2D slices and topological views, first, by looking at their
performance with respect to common function analysis tasks. I also demonstrate
3 usage scenarios: the 2D sinc function, neural network regression, and
optimization traces. Based on this evaluation, I characterize the advantages
and drawbacks of each of these approaches, and show how interaction can be used
to overcome some of the shortcomings. 


I also presented Hypersliceplorer, an algorithm for generating 2D
slices of multi-dimensional shapes defined by a simplical mesh.  Often, slices
are generated by using a parametric form and then constraining parameters to
view the slice. In this case, I developed an algorithm to slice a simplical
mesh of any number of dimensions with a two-dimensional slice. In order to get
a global appreciation of the multi-dimensional object, I show multiple slices
by sampling a number of different slicing points and projecting the slices into
a single view per dimension pair. These slices are shown in an interactive
viewer which can switch between a global view (all slices) and a local view
(single slice). I show how this method can be used to study regular polytopes,
differences between spaces of polynomials, and multi-objective optimization
surfaces. 


Finally, I develop a method for predicting the rendering time to display
multi-dimensional data for the analysis of computer simulations using the
HyperSlice~\cite{Wijk:1993} method with Gaussian process model reconstruction.
My method relies on a theoretical understanding of how the data points are
drawn on slices and then fits the formula to a user's machine using practical
experiments.  I also describe the typical characteristics of data when
analyzing deterministic computer simulations as described by the statistics
community.  I then show the advantage of carefully considering how many data
points can be drawn in real time by proposing two approaches of how this
predictive formula can be used in a real-world system.

