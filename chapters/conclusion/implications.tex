
\section{Implications}

The main goal of my thesis was to explore what is possible with slice-based
visualizations of continuous multi-dimensional datasets. My hope is that this
work will serve as a basis for an increasing focus on direct visualization of
multi-dimensional objects. Often it seems that the default analysis technique
for more than three dimensions is to reduce
the dimensionality of the
data and then render the reduced data on screen. This suffers from issues of
distortion of distances and relative sizes. The analysis tasks for
multi-dimensional data are all developed around understanding the carefully
chosen dimensions. Hence, transforming these dimensions takes away alot of
contextual knowledge about the simulation. 

I also hope to bring more attention to continuous multi-dimensional data analysis.
In the visualization community, most of the work on multi-dimensional and high-dimensional
data has focused on the discrete case. There are many task taxonomies, techniques,
and applications for discrete data. My hope with this thesis is that by developing
a task and data taxonomy as well as an in-depth study of direct visualization
tehcniques will bring similar attention to multi-dimensional continuous data
analysis. There are a number of under-explored application areas in this
field. I have identified some in my own work, but with further research in this
field will bring more knowledge and understanding about how we, as three-dimensional
beings can understand multi-dimensional continuous datasets.




