\section{Motivation}
\label{sec:motivation}

%Visualizations of multi-dimensional continuous spaces is important to
%understanding the behavior of
%simulations~\cite{Piringer:2010,Torsney-Weir:2011} and mathematical
%objects~\cite{Karpenko:2010}. \tm{should really cite Hyperslice here} Most of the work on
%multi-dimensional spaces has been on understanding manifolds. These are
%the result of studying dependent relationships of simulations, for
%example. In other cases we want to study the shapes of objects. \tm{citation?} With
%multi-dimensional shapes, we want to study the connectivity between
%dimensions. \tm{what is that?} Visualization techniques for multi-dimensional shapes are
%not as well-developed. In this work we concentrate on the visualization
%of these shapes via multi-dimensional hulls.

The visual analysis of multiple dimensions is one of the central themes of
visualization research. In principle there are two conceptual types of problems
that amount to two different mental models. (1) Often, the data set is considered to be truly discrete 
%where the generation of
%new points is either too expensive or the underlying structure of the space is
%unknown. This leads to
and projection methods, such as 
scatterplots and dimensionality reduction techniques, are used for its analysis.
Typical examples include business applications, in which one is analyzing
customer data.
The focus of
my work is different in that (2) I am focusing on continuous multi-dimensional data spaces. For computational purposes, the data set is then merely a
set of points sampled from a continuous phenomenon of study. This is
rather common in simulation and engineering applications or for the study of
continuous algorithmic parameters in modeling environments, including machine
learning applications~\cite{Sedlmair:2014}. Of course, for such scenarios, projection based
visualization might be of help as well. However, they do not respect the mental
model of the object of study~\cite{Tory:2004}.

To comprehend these continuous data spaces, I extend the
HyperSlice~\cite{Wijk:1993} method, which presents a number of slices through
data space, all connected through one point in data space (called the focus
point). Slicing has a number of advantages including undistorted views of the
space and the preservation of distances. The disadvantage is that only one
focus point can be shown at a time. Vastly different views of the object may be
seen depending on the location of the focus point. Navigating multiple
dimensions, the user may also lose their place during interaction.  I create a
two-dimensional slice by constraining all but two parameters to the focus point
value.

%our previous work, 
In Sliceplorer~\cite{Torsney-Weir:2017a}, I suggest to
present 1D slices instead of 2D slices in such cases. While a 1D slice carries
less information than a 2D slice, they could now present a global view of the
multidimensional object by over-plotting many 1D slices. This advantage was
worth the loss of 2D information. Here, I take the idea of more global overviews and revisit 2D 
slices for closed multi-dimensional objects, whose overall multi-dimensional shape is
of great importance. This has been motivated by a number of real-world
application scenarios, from comprehension of multi-dimensional polytopes by 
geometers, to applications in computational science, to
multi-dimensional Pareto-front analysis. Polytopes are the generalization of polygons and polyhedra to multiple dimensions. By showing only slices of the outlines of these
polytopes we can again create global views of these data sets through
over-plotting of many focus points.

%This parametric
%description is not always practical, however. Often we only have a point
%cloud or a simplical mesh describing the surface of the object.

I address the issue of selecting a focus point by sampling a number of focus
points and producing projections of 2D slices (\autoref{fig:interface}). This
\emph{global view} gives the user an overview of the behavior of the object
without having to navigate manually. Since we are viewing just
the outer hull of the object, I draw these as a projection of a set of 2D
slices. I use linked highlighting to show all slices for a particular focus
point. In addition, the user can click on a particular slice and switch to the
\emph{local view}. The local view begins by showing the particular slice the
user clicked on. The user can add additional focus points and select particular
slices for comparison. This interaction models Shneiderman's mantra: ``overview
first, filter and zoom, details on demand''~\cite{Shneiderman:1996}.

My major contribution is an algorithm called Hypersliceplorer for computing the
intersection of a 2D slice with a simplical mesh in any dimension (see
\autoref{sec:algorithm} and \autoref{fig:slicing}). The issue is that a 2D
plane does not have a well-defined normal in spaces higher than three.
Therefore, one cannot use the typical point-normal form of a plane in order to
compute the intersection of the plane with the simplex. Instead, I treat the
plane as a point with two free parameters representing the plane. Then, I show
how this representation allows us to compute how a multi-dimensional simplex
intersects a 2D plane. This approach lets one compute slices of a
multi-dimensional object without a parametric form of the surface. I also
demonstrate the results of this algorithm with an interactive interface I
developed.

I evaluate our algorithm and interface in two ways representing their recommended
evaluation methods according to the nested
model~\cite{Munzner:2009}. For the overall technique and
interactive viewer, I demonstrate the effectiveness of Hypersliceplorer by
presenting three case studies. For the underlying algorithm I present an
analysis of the running time. 

In summary, my contributions are:
\begin{itemize}
\item
  an algorithm for computing 2D slices of multi-dimensional polytopes defined by a
  simplical mesh,
\item
  an interactive viewer combining both global and local views of slices,
\item
  three case studies demonstrating the technique,
\item
  and an analysis of the running time of the slicing algorithm.
\end{itemize}

