\section{Interface}
\label{sec:interface}

I developed an interactive viewer to browse and select slices of interest in
order to build up an understanding of the object we are viewing. Slicing is an
inherently interactive operation. Depending on what focus point 
we will see different aspects of the data. In a multi-dimensional
space, it is easy to get lost navigating freely without guidance.
However, if we show all slices at once the user cannot closely examine one
particular aspect of the data.  Thus, the interactive interface, shown in
\autoref{fig:interface}, has two modes: a global view and a local view. The
global view is designed to give an overview of the general shape.  By selecting 
a slice and corresponding focus point of interest, the user can then
switch to the local view and gradually add additional slices at new focus
points.
%implementation?

\subsection{Global view}

The global view (\autoref{fig:interface:global}) gives an overview of the
possible cross sections of the object. By default we show slices for the first
50 focus points sampled using a Sobol sequence~\cite{Sobol:1967}. This has the
advantage of being both space-filling and easy to add additional sample points
if required. Since we are slicing hulls of simplicial meshes, each slice is a contiguous
line plot in the view. We use alpha blending in order to show the distribution
of hull shapes in each pair of dimensions. 
%We did this to show not only want to know which shapes are possible but also the frequency. 
From this the user can get insight into whether or not a shape has a regular 
structure.

With more than one slice one cannot easily tell how the slices correspond 
between panels in the layout. I address this by using linked highlighting
between the plots. If the user mouses over a slice in one plot the slices
corresponding to that focus point are highlighted in the other plots. In 
addition, the user can click on a particular slice of interest to focus in 
on that particular slice. This brings the user into the local view mode.

\subsection{Local view}

The local view mode of the interface allows the user to narrow in on a
particular focus point and then explore how other slices of the figure relate
to that one.  The focus point is represented as a dot projected on each sub
plot. 
%The user can drag this point to another focus point location which will then show the slice corresponding to that focus point.  
The user can change the focus point by dragging the
focus point dot to a new location. Thus, the user can change one or two focus
point values per dragging interaction.  The user can also add additional focus
points by clicking the ``+ fp'' button in the upper left of the interface. Each
focus point is automatically colored based on a discrete color map from
ColorBrewer~\cite{Harrower:2003}. The slices themselves and the focus points
are linked through a similar color.  For example, one mode of exploration this
view supports is examining the faces orthogonal to one of the slices.  The user
can return to the global view by clicking the ``deselect'' button on the top
left of the interface.


