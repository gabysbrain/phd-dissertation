\section{Conclusion and future work}
\label{sec:conclusion}

In this chapter I have presented Hypersliceplorer, an algorithm to compute
2D slices of multi-dimensional shapes defined as a simplical mesh. 
I also discussed an interactive interface we developed to view the slices.
I evaluated our method in two ways: through three target use case scenarios
and with measuring the running time.

In the future I will improve the speed of our algorithm by integrating a
spatial data structure such as a k-d tree or bounding box method. The current
algorithm must find an intersection with every simplex in the figure even if
no intersection is possible.
The data structure will help to avoid these extra checks. This should
improve the speed of the algorithm by avoiding unnecessary intersection tests.

Currently, I have only tested our method with convex hulls of shapes. I
plan to also examine the visualization possibilities with non-convex hulls
and with pre-generated hulls. Perhaps this method will lead to new insights
in mesh generation algorithms.

I will also work closely with target user groups to customize the interface
for their specific goals. For example, geometry users may more integration with
the Schlegel diagram while multi-objective optimization users may need better
support for local neighborhoods.


