\section{Related work}
\label{sec:related-work}

Multi-objective optimization and multi-dimensional objects are two areas
where it is important to study shapes in over three dimensions. We discuss 
these areas below. Topological techniques are based on viewing critical points
of manifolds~\cite{Correa:2011,Gerber:2010} or how contours merge and 
split~\cite{Carr:2003a}. We do not discuss them further. Manifold analysis
is very different than visualizing shapes. 

The need to understand multi-dimensional polytopes is apparent to 
geometers~\cite{Ziegler:2012}.
However, there are a number of cases in computational science where the
understanding of the size and the shape of a sub-section of the parameter space
is of importance~\cite{Bergner:2013,Sedlmair:2014}. One of these cases is
highlighted in \autoref{sec:bernstein}. Another use case is the study
of multi-dimensional Pareto fronts (\autoref{sec:pareto}).

\subsection{Multi-objective optimization}

In multi-objective optimization we have several scalar values that we wish to
optimize. The set of optimal points is known as the Pareto front.
If each objective measure is continuous then we have a continuous hull in one
orthant. We want to use this hull to analyze the trade-offs between objective
measures. Interactive decision maps~\cite{Lotov:2004} show a 3D Pareto front as
a series of 2D slices. Any objectives past three must be constrained to a value
however. Objective functions are difficult to sample since we often do
not have control over the sampling of the range of a function.  To
generate this hull one often samples the objective functions and then computes
the Pareto points using an algorithm such as NSGA-II~\cite{Deb:2002} or the
skyline algorithm~\cite{Borzsony:2001}. We can then generate the hull using
multi-dimensional marching cubes~\cite{Bhaniramka:2000}, the quickhull
algorithm~\cite{Barber:1996}, or alpha shapes~\cite{Edelsbrunner:1983}. These
can then be viewed in Hypersliceplorer as we do in \autoref{sec:pareto}.

An alternative is to treat the samples as a fixed set and then visualize the
relationship between possible combinations of objectives. Typically this is
done by examining the weight space through interaction. 
LineUp~\cite{Gratzl:2013} uses a ranked list approach and shows the user
how rankings will change as the user changes the relative weighting for each
objective. WeightLifter~\cite{Pajer:2016} extends this by also showing the
stability of rankings. The user can understand how much a particular objective
is affected by its weighting. This can help speed interactive exploration. 
Finally, the joint contour net~\cite{Carr:2014} can be used to compute how
often two objectives hold particular values simultaneously. 
In our case, the mental model is a continuous one. Thus it makes more sense
to show a continuous Pareto front.

\subsection{Multi-dimensional objects}

%TM: I wanted to add a sentence like this, but I dont think we need to after all: Since we are focusing on the visualization of continuous objects in multiple dimension, literature on projection based techniques is not relevant and will not be mentioned. 

%We review related work on the visualization of multi-dimensional continuous objects.

When speaking of 3D polytopes, their source is usually either from reconstruction of 3D point clouds 
(see Dey~\cite{Dey:2006})
or from iso-surfacing techniques (see Wenger~\cite{Wenger:2013}).
%However, we believe that the method of choice for visualization of 3D shapes are 3D renderings.
%Visualization of (continuous) objects in three (or fewer) dimensions is not that relevant for our discussion, since we believe that our method will not be of advantage here. Still, in the are of volume visualization there are quite a wealth of approaches for creating an iso-surface, which creates a 3D polytope. See also the excellent text books by Wenger or Day
%
%often falls under the name of
%volume visualization.  Three-dimensional volumes can be rendered using
%techniques isosurface techniques like marching cubes~\cite{Lorensen:1987a} or direct volume visualization. 
There are extensions to iso-surfacing techniques in multiple dimensions~\cite{Bhaniramka:2000}, 
%Marching cubes has been extended to arbitrary
%dimensions 
but in more than three dimensions we must distort the space somehow to visualize the object. 

For the visualization of 4D polytopes, there are a number of techniques for moving from four to three dimensions.  The
Schlegel diagram~\cite{Sommerville:1929} is one such method based on
projection. We pick a face of the figure, usually the largest, which is a
three-dimensional object. Then, all other faces are ``packed'' inside this face in
such a way that we can show the connections between faces. The Schlegel diagram
works well for regular polytopes where we have some previous intuition about
the faces. However, for an arbitrary simplical mesh, any face is a simplex
which we need to project into. All Schlegel diagrams of a simplical mesh look
like a simplex with a number of other simplices inside them. It can be
difficult to recover what the original object looks like because the
cross section is lost. An alternative approach is to treat the fourth dimension
as time and then produce an animation of the evolution of the shape in three
dimensions. In this case each frame of the animation is a 3D slice of the 
object. Rather than first projecting from 4D to 3D and then rendering the
projection, Hanson and Cross~\cite{Hanson:1993} propose a method to first
render the object in 4D and then view the three-dimensional projection. This
allows them to show unique lighting effects from the 4D surfaces.
%One can also use a stererographic projection. 
As with all projection methods, if the user is unaware of the details of the
method it can be difficult to build a mental model of the shape under study.

Hasse diagrams~\cite{Battista:1988} are based on showing the connectivity
between vertices of an object. These can be seen as network diagrams where the
vertices of the figure are the nodes in the graph and the edges of the graph
are the edges in the figure. These have a number of layout issues.  For visual
understanding, humans prefer a 2D planar graph~\cite{Kieffer:2016}. Good layouts
of the Hasse diagram must balance human aesthetic needs like few edge crossings
with the geometric interpretation. 
There are automatic layout
algorithms, such as the one by Battista et al.~\cite{Battista:1988}, but these
do not work in all cases.

For more than four dimensions, 
projection methods no longer work as well.  Techniques based on slicing the
space can be extended to any number of dimensions.  The techniques to perform
this so far, such as HyperSlice~\cite{Wijk:1993},
HyperMoVal~\cite{Piringer:2010}, and Sliceplorer~\cite{Torsney-Weir:2017a},
are designed to show slices of multi-dimensional manifolds.
They produce slices by
constraining all but two (for 2D slices) or one (for 1D slices) of the
dimensions to the focus point value and then producing a heatmap, contour plot,
or function plot. Sliceplorer addressed the focus point issue by sampling over
a number of focus points and projecting them down.  Exploded view
diagrams~\cite{Karpenko:2010} offer a hybrid method between a 3D volume
visualization and slicing.  However, they 
%also require a parametric description of the object under study and 
are limited to 3D objects. 
The global view of Hypersliceplorer is inspired by the idea of examining
cross sections. We also have a local
view which permits the user to look at a small number of self-selected slices.
We have developed a method to produce slices based on a simplical mesh which
is very useful given a discretized surface (see \autoref{fig:slicing}). 

