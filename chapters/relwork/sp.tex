\section{Related Work}\label{sec:background}

The question of how to comprehend multi-dimensional data is a heavily
researched area in visualization. There are two principal approaches,
projection techniques (such as scatterplots) and slicing techniques (such as
HyperSlice). Projection techniques generally show all of the data and,
therefore, represent a more global view. On the other hand, slicing techniques
present more of a local view around a point of interest (which we call focus
point). The lion's share of previous work is concerned with projection
approaches of discrete data. In contrast, we are focusing on continuous
multi-dimensional scalar functions and seek to combine the strengths of
projection and slicing. Given our focus on continuous data, our options for a
visual exploration are limited and could be categorized into three areas (a)
discretization, (b) local methods, (c) global methods.

\subsection{Discretization}

There is a number of different approaches to display discretized
multi-dimensional functions. The typical approaches are scatterplot
matrices~\cite{Hartigan:1975}, parallel coordinates~\cite{Inselberg:1985}, star
coordinates~\cite{Kandogan:2000}, and RadVis~\cite{Hoffman:1997}. 
Star coordinates and RadVis were generalized into one framework by Lehmann
and Theisel~\cite{Lehmann:2016a}.
These can all be combined with a variety of dimensionality
reduction techniques~\cite{Holbrey:2006}. However, all of these seem
inappropriate if the mental model of the function we are studying is a continuous
one. In such a case all of these projection techniques would fail to properly
convey the complexity of the underlying continuous phenomenon. Hence, while
discretization seems like an easy way out, it is not a proper alternative for
studying \emph{continuous} multi-dimensional functions, such as regression
functions or classification boundaries. Here, one of the main concerns is
understanding a continuous phenomenon and a good visualization design should
thus respect this mental model~\cite{Tory:2004a,Sedlmair:2012,Liu:2010a}.
%We have argued in the past (and it is well accepted) that one of the main
%concerns of a proper visualization design is to respect the mental model of
%the user~\cite{Tory2004b,Sedlmair2012a}.

%The reason that we are not using discrete techniques is because in domains like
%regression or parameter space analysis the user's mental model is in the
%continuous realm. Input parameters are continuous and the output is also
%continuous \msnote{really, do we have some reference to back this claim up? I think it is a good point but needs more backing up ... wither through a ref or through careful argumentation (maybe one more sentence or so)}. According to ???~\cite{???} we should match the visualization with
%the user's mental model.  Continuous data is not devoid of research though.

\subsection{Local methods}

The idea of a local technique is to focus on a part of the function.
Interaction is used to explore other parts of the function. One of the oldest
approaches here is the HyperSlice technique by van Wijk and van
Liere~\cite{Wijk:1993}.  HyperSlice is a technique where the function is shown
directly but in multiple 2D slices laid out similar to a scatterplot matrix. In
many ways, our work was inspired by this work. One of the drawbacks of
HyperSlice is that one has to choose a focus point --- a point common to all 2D
slices. Exploring the full data set then shifts over to exploring all possible
focus points. Although not created for HyperSlice specifically,  techniques
like the grand tour~\cite{Asimov:1985}, projection pursuit~\cite{Huber:1985},
and optimal sets of projections~\cite{Lehmann:2015b} might be appropriate to
tackle this issue.
%\ttwnote{I'm not sure we need to cite projection persuit, etc
%here since it doesn't work on continuous data}\msnote{I would cur except the Lehman paper is
%one that they explicitly asked to include}. Torsten: we say "might" and I think it is ok.
All of these approaches are
still local though. A mental image of the global function can only be built up
over time and with mental effort by browsing through the focus points. Our
approach seeks to overcome this limitation while keeping the benefits of ease
of understanding.  

Note that for some tasks a local view might in fact suffice, such as when one is interested in the robustness of an extrema value. 
For example, Tuner~\cite{Torsney-Weir:2011} used 2D
continuous slices, letting the user navigate them via selecting Pareto optimal focus points in a separate view. 
%Tuner used a separate view to show the uncertainty, but could then show the prediction uncertainty for all parameter values at once.
%On the other hand, 
Berger et
al.~\cite{Berger:2011} use coordinated views of
scatterplots and parallel coordinates to show additional (continuous) prediction uncertainty. Having a discrete approach provided extra space to display information about the prediction uncertainty for the currently selected point.

\subsection{Global methods}

The visualization community has developed many global views of
multi-dimensional continuous functions. Continuous
scatterplots~\cite{Bachthaler:2008} and continuous parallel
coordinates~\cite{Heinrich:2009} can encode a multi-dimensional density field
into either two dimensions or more than two dimensions respectively.  Here, we
are not as concerned with the density field, rather we are concerned with the
manifold created from a continuous multi-dimensional scalar function. 

Topological methods like topological
spines~\cite{Correa:2011}, the work by Gerber et al.~\cite{Gerber:2010}, and
contour trees~\cite{Carr:2003a} extract extrema and saddle points from a
function and then show these. These methods are good for seeing the relative
relation of extrema in a function. However, they do not work for important
tasks like robust optimization. Here, one does not necessarily want to find the
global optimum but wants an optimum in a relatively ``flat'' area of the
parameter space. 
%There is also multi-field 
%topology~\cite{Duke:2012,Huettenberger:2014,Carr:2015} which we do not consider
%in this paper since we are concerned only with \emph{scalar} functions\ttwnote{is this definition clear? i.e. scalar means 1D range}.

In our case, we use line plots
%which
%any student of high school algebra has had to study. We use the
together with the widely-used technique of projection to overdraw 1D slices. This approach is similar to the work by Hall et al.~\cite{Hall:2014} but differs in two major ways. (1) They showed 2 primary dimensions with slices and then used the third for color limiting their view to three dimensions and (2) they were concerned with isosurface extraction. Our technique can scale to any number of dimensions and we evaluate based on a much broader set of tasks and applications, such as parameter space analysis. 
%We also compare our technique to curve boxplots~\cite{Mirzargar:2014}, an extension of contour 
%boxplots~\cite{Whitaker:2013} to arbitrary curves. In our evaluation (see
%\autoref{sec:task-eval}) we find that only showing the distribution can hide
%some sutulties of the function that the user may want to see.

%Rather than looking at the continuous function directly we could simply
%sample the continuous function and then use the wealth of discrete
%visualization techniques that have been developed over the years.
%Discrete views by and large leave a lot of room for showing additional
%visual channels. For example, Berger et
%al.~\cite{Berger:2011} use coordinated views of
%scatterplots and parallel coordinates to show prediction uncertainty
%information. The discrete approach left them with additional room to
%display information about the prediction uncertainty for the currently
%selected point. Tuner~\cite{Torsney-Weir:2011} used 2D
%continuous slices so had to use a separate view to show the uncertainty,
%but could then show the prediction uncertainty for all parameter values
%at once. Researchers have developed task
%taxonomies~\cite{Brehmer:2014,Amar:2005} to abstract the
%common tasks that visualization users want to perform. By and large,
%these taxonomies were developed with discrete data in mind and do not
%even use examples from
%
%There has been alot of work done by the visualization community on
%visualization of multi-dimensional \emph{discrete} data. Naturally, one
%could sample the continuous function and then use the wealth of discrete
%visualization techniques like scatterplots and bar graphs. In addition,
%there have been a number of techniques developed specifically with
%viewing multi-dimensional data in mind like parallel coordinate
%views~\cite{Inselberg:1985}, scatterplot
%matrices~\cite{SPLOM}, and \ttwnote{something}. By and large
%discrete data visualizations can be considered different projection
%techniques\ttwnote{projection vs slicing}. For example, Berger et
%al.~\cite{Berger:2011} use coordinated views of
%scatterplots and parallel coordinates to show prediction uncertainty
%information. The discrete approach left them with additional room to
%display information about the prediction uncertainty for the currently
%selected point. \msnote{this paragraph is write repetitive to the last one. Also, for both I'm missing somewhat to context on why they are actually discussed and how exactly they relate to our work}
%
%Rather than choosing a viewpoint there are also a wide variety of techniques to
%reduce the data before viewing it. For example, dimension reduction techniques
%are often used on a discrete data set to reduce the number of attributes that
%we show. These include linear techniques like principal component
%analysis~\cite{PCA} and linear descriminant analysis~\cite{LDA}. Other
%techniques work similar to dimension reduction where they try to find
%projection axes but instead work on the view axes. Techniques like the grand
%tour~\cite{Asimov:1985} and projection persuit~\cite{Huber:1985} fall in here.


%\msnote{I moved the entire task part to below, it does not fit the flow very well here, and we can save space if we simply discuss it in section 4}
%\subsection{Task taxonomies}
%
%To help us reason about when to use which visualization techniques, 
%researchers have developed
%low-level~\cite{Amar:2005} and high-level task
%taxonomies~\cite{Heer:2012}. Brehmer and Munzner~\cite{brehmer:2013} link
%these together into a comprehensive multi-level typology. These
%taxonomies can be applied to continuous data but not directly. As we
%will see in \autoref{sec:evaluation}, these taxonomies need to be adapted for
%continuous data.
%%Researchers have also done perceptual studies to
%%determine what are the best visual encodings for certain tasks and data
%%characteristics. To our knowledge, these were mostly done with discrete
%%data in mind. For example, Correll and
%%Gleicher~\cite{???} \ttwnote{did something}. \msnote{perceptual studies need to be better connected to the taxonomies point ... in general, I would recommend to have one sentence at the beginning of each paragraph that states the main point ... e.g. here something along the lines of there is tons of work in projections of discrete data, while there is much less on "projection" in continuous data ... that's btw also a good idea for all other paragraphs}
%%\ttwnote{maybe cite mackinlay too?} 
%Sedlmair et al.~\cite{Sedlmair:2014} started to enumerate the tasks one would
%do with multi-dimensional continuous data in a parameter space analysis
%setting. They did not differentiate between continuous and discrete views of
%the parameter space in their taxonomy, however.
%Part of what we are advocating in
%this work is more consideration of multi-dimensional continuous data as
%a first class citizen when developing task hierarchies.
%and perceptual studies. %\msnote{Nice point!}
%MS: taking out this point, it comes out of nowhere


%\ttwnote{mathematical surfaces paper} \ttwnote{contour boxplots evolved into curve boxplots...}
%\ttwnote{color map creation?}
%\ttwnote{some discretization is helpful, but how much?}
%
 %\msnote{needs to be rephrased, it sounds much too defensive ... unfortunately, Vis reviewers are not appreciating such an honest statement in my experience}

%\msnote{(1) it seems a bit disorganized at the moment,
%i.e. missing a clear train of thought that is connected to our idea.
%A dedicated into paragraph would help saying what will happen
%in this section and why. Maybe we can even have subsections then.
%For each block of related work we should add a clear statement 
%how this is different from what we are doing. For some, this is already there
%for others not.}
%
%\msnote{(2) to increase the flow it would also be good to think about the “glue”
%between the different related aspects (paragraphs and/or subsections).
%At the moment, it feels a bit disconnected.}
%
%\msnote{(3) I think there are already some nice text blocks in there. 
%Others need to be improved. In general, I felt that end it gets much stronger ...}
%
%\ttwnote{need to say something about volume rendering here...}
