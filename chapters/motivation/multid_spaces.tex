
\section{Multi-dimensional spaces}
\label{sec:motivation:multi-d}

There are a number of varied domains where one can apply the analysis of
continuous multi-dimensional data.
\ttwnote{contribution: taxonomy of domains}

%Often work on continuous multi-dimensional data analysis is done in the context
%of design studies~\cite{Sedlmair:2012}. The issue is transferability, while
%important, is not always considered. 

%Each of these application scenarios are often treated individually. These all
%fit under a unified visualizaiton method though.

These domains can be broadly classified into two types based on their analysis
goals. One type, \emph{manifold} analysis, deals with understanding the
relationship between inputs and outputs. This is a functional relationship.
The user wants to inspect how changes in the inputs (independent variables)
affect the outputs (dependent variables). One can also perform \emph{shape}
analysis. Here, in terms of the analysis goals, there is no identification of
independent and depdendent variables. \ttwnote{expand this a bit}

\subsection{Manifolds}
\label{sec:manifolds}

Studying manifolds means studying functional relationships. The critical issue
is understanding the relationship between independent and dependent factors.
\ttwnote{subtasks}. Analyzing functions is of course within the realm of
manifold analysis. \ttwnote{examples of work} Another area where understanding
the manifold is important is analyzing optimization surfaces and functions.
\ttwnote{want to see number and relative depth of bowls/peaks same as sens
analysis} \ttwnote{algo can get stuck}

\ttwnote{example} Simulation analysis is another area where the analysis tasks,
in the abstract, is examining functions. \ttwnote{latex equation here?} If we
look back at the weather simulation from before, the inputs to the function are
things like the temperature and pressure. The output is, for example, the
likelihood of rain the next day. The function is the simulation itself.
Computer simulations are deterministic. Stochastic simulations can be converted
to deterministic by treating the random number generator seed as another
parameter.  A deterministic simulation has a mapping from each unique input
parameter configuration to an output value. This is the same as a functional
relationship. The sensitivity, extrema, and \ttwnote{something} are also
important to simulation analysis. \ttwnote{what is the goal?}
Thus, these can all be analyzed with similar visualizations.

\subsection{Shapes}
\label{sec:shapes}

One may also want to understand the relationship or correlation between
multiple continuous values. This is in contrast to studying the relationship
between independent and dependent variables in the manifold case. In this case
we want to study the relationship of all variables. Careful study of the shape
of the dataset can give insight into the overall shape of the object. For
example, one may want to know if the multi-dimensional object is shaped like a
sphere, donut, or box.
In addition one may be interested in any kinks or cusps in the dataset.
\ttwnote{why?} These do not necessarily need to be true cusps. Changes in
the gradient and curvature of the shape are also of great interest.
%Detecting these changes can be difficult to do numerically so 

These analysis tasks of multi-dimensional shapes can be applied to a number of
different areas. The study of polytopes is one such area and perhaps the most
direct application of understanding multi-dimensional continuous shapes.
Polytopes are the multi-dimensional generalization of polyhedra and polygons.
The tasks are to understand the symmetries and patterns making up the
polytopes~\cite{Ziegler:2012}. Perhaps a less obvious connection is the
analysis of the tradeoff curves in multi-objective optimization. Since we are
performing optimization we are interested in the tradeoffs amongst all the
non-dominated points~\cite{nondominated}. This is also known as the Pareto
front. In this case, the user wants to understand what are the costs of
reducing one or more parameters in order to increase the value of others. Cusps
or large changes in curvature in these datasets are important since they show
fundamental changes in the rate of tradeoff.  With a proper view of a
multi-dimensional object we can also view differences betweeen two objects
directly. With 1-, 2-, or 3-D objects we show these differences directly and
can understand them.  \ttwnote{showing a difference object is hard enough}
However, the two-stage processing of showing a difference object and then
performing dimension reduction makes it difficult to view difference objects
with dimension reduction. Slicing, as a direct visualization technique, does
not have this problem.

\ttwnote{conclude, move on}

