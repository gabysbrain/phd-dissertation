
\section{Multi-dimensional spaces}
\label{sec:motivation:multi-d}

There are a number of varied domains where one can apply the analysis of
continuous multi-dimensional data.
\ttwnote{contribution: taxonomy of domains}

%Often work on continuous multi-dimensional data analysis is done in the context
%of design studies~\cite{Sedlmair:2012}. The issue is transferability, while
%important, is not always considered. 

%Each of these application scenarios are often treated individually. These all
%fit under a unified visualizaiton method though.

These domains can be broadly classified into two types based on their analysis
goals. One type, \emph{manifold} analysis, deals with understanding the
relationship between inputs and outputs. This is a functional relationship.
The user wants to inspect how changes in the inputs (independent variables)
affect the outputs (dependent variables). One can also perform \emph{shape}
analysis. Here, in terms of the analysis goals, there is no identification of
independent and depdendent variables. \ttwnote{expand this a bit}

\subsection{Manifolds}
\label{sec:manifolds}

Studying manifolds means studying functional relationships. The critical issue
is understanding the relationship between independent and dependent factors.
\ttwnote{subtasks}. Analyzing functions is of course within the realm of
manifold analysis. \ttwnote{examples of work} Another area where understanding
the manifold is important is analyzing optimization surfaces and functions.
\ttwnote{want to see number and relative depth of bowls/peaks same as sens
analysis} \ttwnote{algo can get stuck}

\ttwnote{example} Simulation analysis is another area where the analysis tasks,
in the abstract, is examining functions. \ttwnote{latex equation here?} If we
look back at the weather simulation from before, the inputs to the function are
things like the temperature and pressure. The output is, for example, the
likelihood of rain the next day. The function is the simulation itself.
Computer simulations are deterministic. Stochastic simulations can be converted
to deterministic by treating the random number generator seed as another
parameter.  A deterministic simulation has a mapping from each unique input
parameter configuration to an output value. This is the same as a functional
relationship. The sensitivity, extrema, and \ttwnote{something} are also
important to simulation analysis. \ttwnote{what is the goal?}
Thus, these can all be analyzed with similar visualizations.

\subsection{Shapes}
\label{sec:shapes}



\begin{itemize}
\item
  geometry
\item
  Pareto analysis
\item
  spaces of polynomials
\item
  isosurfaces
\item
  spheres
\end{itemize}

