\section{Multi-dimensional spaces}
\label{multi-dimensional-spaces}

There are a number of varied domains where one can apply the analysis of
continuous multi-dimensional data.
\ttwnote{contribution: taxonomy of domains}

%Often work on continuous multi-dimensional data analysis is done in the context
%of design studies~\cite{Sedlmair:2012}. The issue is transferability, while
%important, is not always considered. 

%Each of these application scenarios are often treated individually. These all
%fit under a unified visualizaiton method though.

These domains can be broadly classified into two types of analysis.  One type
of analysis, \emph{manifold} analysis, deals with understanding the
relationship between inputs and outputs. This is a functional relationship.
The user wants to inspect how changes in the inputs (independent variables)
affect the outputs (dependent variables). One can also perform \emph{shape}
analysis. Here, in terms of the analysis goals, there is no identification of 
independent and depdendent variables. 
\ttwnote{expand this a bit}

\ttwnote{also talk about shapes vs manifolds}

\subsection{Manifolds}
\label{sec:manifolds}

    \begin{itemize}
    \item
      functions
    \item
      simulations
    \item
      optimization problems
    \end{itemize}

\subsection{Shapes}
\label{sec:shapes}

    \begin{itemize}
    \item
      geometry
    \item
      Pareto analysis
    \item
      spaces of polynomials
    \item
      spheres
    \end{itemize}

