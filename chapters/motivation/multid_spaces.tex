
\section{Multi-dimensional spaces}
\label{sec:motivation:multi-d}

There are a number of domains where one can apply the analysis of continuous
multi-dimensional data.  As of yet, there has not been a comprehensive data and
task analysis for multi-dimensional continuous data analysis. For discrete
data, there are several task 
analyses~\cite{Shneiderman:1996,Brehmer:2013,Amar:2004}. 
However, they are
focused on identifying and selecting particular data items. Continuous datasets
consist of ranges of values as well as functions. Functions can be seen as a
mapping from ranges of numbers to other ranges. The analysis task here is to
study these ranges.  Tasks addressing studying the mappings or studying the
relationship between ranges have not yet been covered by visualization task
analyses. Thus, there is no comprehensive source for what analysis tasks one
wants to perform given a continuous multi-dimensional dataset.  Work in this
area has traditionally focused on developing a specific visualization for a
specific task. For example, topological spines extracts critical points from a
scalar field~\cite{Correa:2011}. One goal of this thesis is to develop this
task taxonomy for visualization of continuous multi-dimensional data.

%Often work on continuous multi-dimensional data analysis is done in the context
%of design studies~\cite{Sedlmair:2012}. The issue is transferability, while
%important, is not always considered. 

%Each of these application scenarios are often treated individually. These all
%fit under a unified visualizaiton method though.

These domains can be broadly classified into two types based on their analysis
tasks. One type, \emph{manifold} analysis, deals with understanding the
relationship between inputs and outputs. This is a functional relationship.
The user wants to inspect how changes in the inputs (independent variables)
affect the outputs (dependent variables). One can also perform \emph{shape}
analysis. Here, in terms of the analysis tasks, there is no identification of
independent and depdendent variables. We look at each of these in turn.

\subsection{Manifolds}
\label{sec:manifolds}

Studying the mapping between continuous ranges means studying functional
relationships and thus manifolds.  The critical issue is understanding the
relationship between independent and dependent variables.  Subtasks in manifold
analysis include examining critical points, assessing the sensitivity of
parameters, and understanding the shape of the manifold.  Analyzing functions
is of course within the realm of manifold analysis.  Another area where
understanding the manifold is important is analyzing optimization surfaces and
functions.  In this case, the identification of extrema is important for
understanding how many and the relative location of local optima. In addition,
we want to understand the degree to which these are extrema. These can result
in global optimization algorithms ``getting stuck'' in local optima rather than
continuing to search for the global optimum. Optimization algorithms need to be
carefully tuned to properly detect these features and ignore them where
necessary.

Simulations can be used to run experiments that are impractical or impossible
in the real world.  Simulation analysis is another area where the analysis
tasks, in the abstract, is examining functions. If we look back at the weather
simulation from before, the inputs to the function are things like the
temperature and pressure.  The output is, for example, the likelihood of rain
the next day. The function is the simulation itself. Computer simulations are
deterministic.  A deterministic simulation has a mapping from each unique input
parameter configuration to an output value. This is the same as a functional
relationship. The sensitivity and extrema are also important to simulation
analysis.  Thus, these can all be analyzed with visualizations of a manifold.

%Some computer simulations are considered stochastic. That is, their output
%is the determined by a random number generator as well as input values.
%Stochastic simulations can be converted to deterministic by treating the 
%random number generator seed as another parameter.

To date, manifold visualizations have concentrated on a particular analysis
task or a particular application domain. For example, visualizations of the
Morse-Smale complex~\cite{MSvis} are focused on showing only critical points of
the manifold. As with any visualization designed for a specific task, they must
be used in combination with other views for visual analysis of domain-specific
data. Domain-specific visualizations often used linked views to show different
aspects of data to accomplish multiple tasks at once. However, they are purpose
built for a specific domain. While techniques may transfer from one domain to
another~\cite{Seldmair:2012}, it is not always clear how.
My goal is to unify these methods to a certain extent.
As I will show in \autoref{chp:sliceplorer}, slice-based views of manifolds
can be used for a wide variety of tasks in a wide number of domains.

\subsection{Shapes}
\label{sec:shapes}

One may also want to understand the relationship or correlation between
multiple continuous values. This is in contrast to studying the relationship
between independent and dependent variables in the manifold case. In this case
we want to study the relationship of all variables. Careful study of the shape
of the dataset can give insight into the relationship between the various
ranges of dimensions of the object. For example, one may want to know if the
overall shape is a sphere, donut, or box. In addition one may be interested in
any kinks or cusps in the dataset. \ttwnote{why?} These do not necessarily need
to be true cusps. Changes in the gradient and curvature of the shape are also
of great interest.

These analysis tasks of multi-dimensional shapes can be applied to a number of
different areas. The study of polytopes is one such area and perhaps the most
direct application of understanding multi-dimensional continuous shapes.
Polytopes are the multi-dimensional generalization of polyhedra and polygons.
The tasks are to understand the symmetries and patterns making up the
polytopes~\cite{Ziegler:2012}. Perhaps a less obvious connection is the
analysis of the tradeoff curves in multi-objective optimization. Since we are
performing optimization we are interested in the tradeoffs amongst all the
non-dominated points~\cite{Kung:1975}. This is also known as the Pareto
front. In this case, the user wants to understand what are the costs of
reducing one or more parameters in order to increase the value of others. Cusps
or large changes in curvature in these datasets are important since they show
fundamental changes in the rate of tradeoff.  With a proper view of a
multi-dimensional object we can also view differences betweeen two objects
directly. With 1-, 2-, or 3-D objects we show these differences directly and
can understand them.  Showing a difference object requires the user to reconstruct
the two objects at once. This requires quite a lot of mental processing.
However, the two-stage processing of showing a difference object and then
performing dimension reduction makes it difficult to view difference objects
with dimension reduction. Slicing, as a direct visualization technique, does
not have this problem. \ttwnote{confusing, use images/diagrams to explain?}

\ttwnote{conclude, move on}

