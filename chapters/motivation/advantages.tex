\section{advantages of slicing}
\label{sec:slicing-advantages}

Slicing offers a number of advantages over other multi-dimensional
visualization techniques. 

Slicing is a direct visualization of the multi-dimensional object. In contrast
to methods like projection or feature synthesis, slicing does not distort the
dimensions in order to display them on a two-dimensional screen. Since there is
no distortion, distances in the visual representation are directly proportional
to distances in the object. This is one of the reasons that slicing is popular
in the medical imaging community. Sizes of organs or tumors can be measured
visually on screen. Additionally, relative sizes correspond to what a doctor
would expect to see in the body. Multi-dimensional data is abstract. It lacks
the correspondence to real-world objects that the medical community uses to
understand their two- and three-dimensional datasets.  Nevertheless, it is
still important in multi-dimensional data analysis to properly estimate
distances and relative sizes. \ttwnote{why?}

Another benefit of the direct visualization is that users do not require
extensive training to understand the visualization. \ttwnote{example?} 
\ttwnote{more}

Slice views use the horizontal and vertical axes for showing the effects due to
the input parameters. These axes are the most perceptually uniform~\cite{ref}
and are considered the most effective (\autoref{tbl:enc_effectiveness}).
\ttwnote{expand}

\ttwnote{issues of interaction}


\ttwnote{also talk about shapes vs manifolds}



