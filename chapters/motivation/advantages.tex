\section{advantages of slicing}
\label{sec:slicing-advantages}

Slicing offers a number of advantages over other multi-dimensional
visualization techniques. 

Slicing is a direct visualization of the multi-dimensional object. In contrast
to methods like projection or feature synthesis, slicing does not distort the
dimensions in order to display them on a two-dimensional screen. Since there is
no distortion, distances in the visual representation are directly proportional
to distances in the object. This is one of the reasons that slicing is popular
in the medical imaging community. Sizes of organs or tumors can be measured
visually on screen. Additionally, relative sizes correspond to what a doctor
would expect to see in the body. Multi-dimensional data is abstract. It lacks
the correspondence to real-world objects that the medical community uses to
understand their two- and three-dimensional datasets.  Nevertheless, it is
still important in multi-dimensional data analysis to properly estimate
distances and relative sizes. \ttwnote{why?}

Another benefit of the direct visualization is that users do not require
extensive training to understand the visualization. \ttwnote{example?} 
\ttwnote{more}

Slice views use the horizontal and vertical axes for showing the effects due to
the input parameters. These axes are the most perceptually uniform~\cite{ref}
and are considered the most effective (\autoref{tbl:enc_effectiveness}).
\ttwnote{expand}
One- or two-dimensional plots are replicated for each combination of dimensions
in order to show more than two dimensions at once. This promotes familiarity
of the visualization. Once the user has learned to read a single panel, they
can apply this knowledge to the remaining panels. This approach follows the
principal of small multiples~\cite{ref}.

In order to produce a slice plot one needs to first pick a particular
\emph{focus point} in the multi-dimensional space. This focus point determines
which slices are being viewed. Selecting a good focus point a-priori is
difficult. It either requires a great deal of luck or careful analysis of the
dataset. This is not always possible. Slice-based views require some sort of
interactive focus point selection. Interactively browsing through the slices
requires interaction controls to give the user control over the focus point.
Furthermore, we need some kind of navigation map to show which focus points the
user has selected so that they do not become lost. Neither of these navigation
aids are well developed at this point. This need for interaction is likely one
of the reasons that slice-based views have not developed as much as projection
or topological techniques. Static views are much easier to include in papers
and don't require explanation prior to use.

The other implementation issue of slice-based views is ensuring that the
visualization can remain interactive. In this case, interactive is defined as
the rate at which the user can maintain their \ttwnote{mental
focus}~\cite{Shneiderman:1987}. This is often defined at 10 frames per second.
It can be difficult to compute a 2D slice of an arbitrary complex
multi-dimensional object. 

In my own work, I address these two critical issues. The navigation issue can
be solved with sampling. With both Sliceplorer (\autoref{chp:sliceplorer}) and
HyperSliceplorer (\autoref{chp:hypersliceplorer}), I sample over a number of
focus points and display them all at once. In \autoref{chp:rendering}, I
discuss how we can take advantage of the multi-dimensional geometry to allow
interactive-speed browsing of the focus points. My goal is to highlight the
advantages of slice-based methods for multi-dimensional data analysis. At the
same time I want to address the limitations. The end goal is to bring
slice-based views into the standard toolbox of visualizations.




