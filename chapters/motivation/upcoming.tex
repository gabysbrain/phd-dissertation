
\section{Upcoming}
\label{sec:thesis_outline}

My goal is to highlight the advantages of slice-based methods for
multi-dimensional data analysis. At the same time I want to address the
limitations. The end goal is to bring slice-based views into the standard
toolbox of visualizations.

In \autoref{chp:sliceplorer}, I examine how to visualize manifolds using
slices. I present sliceplorer which is a system to view one-dimensional slices
of multi-dimensional manifolds. I use projections of these one-dimensional
slices instead of showing one focus point at a time. I also go into detail
about the tasks one wants to perform with manifold analysis.

\autoref{chp:hypersliceplorer} extends the idea of projections of slices to the
second dimension. I show how this can be used to effectively visually analyze
the shape of multi-dimensional data. In many cases, this data is given as a
simplical mesh. I introduce an algorithm to compute 2D slices of this mesh.
Using this algorithm, I show how we can visually understand datasets like
Pareto fronts or polytopes.

Finally, in \autoref{chp:rendering}, I discuss how we can take advantage of the
multi-dimensional geometry and GPU architecture to allow interactive-speed
browsing of the focus points. In addition to this algorithm, I develop a method
that can predict the amount of time needed per element to draw one frame of the
visualization. I then show how this estimation formula can be calibrated to a
particular user's hardware.

