
\section{Slices}
\label{sec:slicing-advantages}

Slicing offers a number of advantages over other multi-dimensional
visualization techniques. Slicing is a direct visualization of the
multi-dimensional object. In contrast to methods like projection or dimension
reduction, slicing does not distort the dimensions in order to display them on
a two-dimensional screen. Since there is no distortion, distances in the visual
representation are directly proportional to distances in the object. This is
one of the reasons that slicing is popular in the medical imaging community.
Sizes of organs or tumors can be measured visually on screen. Additionally,
relative sizes correspond to what a doctor would expect to see in the body.
Multi-dimensional data is abstract. It lacks the correspondence to real-world
objects that the medical community uses to understand their two- and
three-dimensional datasets. Nevertheless, it is still important in
multi-dimensional data analysis to properly estimate distances and relative
sizes. 

Another benefit of the direct visualization is that users do not require
extensive training to understand the visualization. The concept of slicing
through a three-dimensional object is a familiar one. Humans are used to this
even from slicing fruits and vegetables with a knife. This concept of slicing
can be extended from this well-known metaphor to cover multiple slices of
multi-dimensional objects.

Slice views use the horizontal and vertical axes for showing the effects due to
the input parameters. These axes are the most perceptually
uniform~\cite{Stevens:1957} and are considered the most effective
(\autoref{tbl:visual_encodings}). One- or two-dimensional
plots are replicated for each combination of dimensions in order to show more
than two dimensions at once. This promotes familiarity of the visualization.
Once the user has learned to read a single panel, they can apply this knowledge
to the remaining panels. This approach follows the principal of small
multiples~\cite{Archambault:2011}. 
%\ttwnote{maybe move this to a section on general vis techniques}

In order to produce a slice plot one needs to first pick a particular
\emph{focus point} in the multi-dimensional space. This focus point determines
which slices are being viewed. Selecting a good focus point a-priori is
difficult. 
%\ttwnote{from michael: this is a challenge again. imho it might be clearer to list all challenges in the same spot. I always liked the spectrum between toplogical approaches and Hyperslices. Then you can say that your goal is to combine their benefits and reduce their weaknesses and how you address these challenges.}
It either requires a great deal of luck or careful analysis of the
dataset. This is not always possible. Slice-based views require some sort of
interactive focus point selection. Interactively browsing through the slices
requires interaction controls to give the user control over the focus point.
Furthermore, we need some kind of navigation map to show which focus points the
user has selected so that they do not become lost. Neither of these navigation
aids are well developed at this point. This need for interaction is likely one
of the reasons that slice-based views have not developed as much as projection
or topological techniques. Static views are much easier to include in papers
and don't require explanation prior to use.

The other implementation issue of slice-based views is ensuring that the
visualization can remain interactive. In this case, interactive is defined as
the rate at which the user can maintain their
concentration~\cite{Shneiderman:1987}. This is often defined at 10 frames per
second. It can be difficult to compute a 2D slice of an arbitrary complex
multi-dimensional object. 




