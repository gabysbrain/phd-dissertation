
\section{Challenges with multi-dimensional spaces}
\label{sec:multi-d-challenges}

Given the importantance of multi-dimensional data analysis, a number of
projects from the visualization community, as well as others, have worked on
this problem. 

\ttwnote{challenges in general}\ttwnote{understandability of more than 3D}
%\item understandability of \textgreater{}3D

Understanding multi-dimensional space is difficult. As humans, we simply do 
not have the spatial analogs in more than three dimensions. A number of methods
have been developed to extract specific features from the multi-dimensional
object. For example, \ttwnote{something about polytopes}, \ttwnote{others}.
However, these only produce an answer without sufficient context. They do not 
necessarily give any intuition as to how to transfer our three-dimensional
knowledge to multi-dimensional spaces.

\ttwnote{how visualization can help}
%\item how visualization can help

\ttwnote{some sort of insight generation from context?}

The difficulty of visualizing a continuous multi-dimensional space on a
two-dimensional screen brings a number of challenges. We treat each dimension
separately, thus, we need several different visual channels. The ranking of
effectiveness of visual channels was proposed by Bertin~\cite{Bertin:1967} and
confirmed through experiments by Mackinlay~\cite{Mackinlay:1986} and
Heer~\cite{Heer} \ttwnote{others?}. We are also limited in how many channels we
can use simulataneously. According to Ware~\cite{Ware}, many channels, such as
red and green are not visually seperable. Furthermore, each dimension of the
multi-dimensional object under study is often treated equally. For example, no
particular axis of a polytope is more important than any other.  We should
encode each dimension using equally weighted effectiveness channels.  With
fewer channels available than data dimensions we either need to reduce the data
or use multiple views to properly visualize the data.

\ttwnote{table/image of visual encoding effectiveness}
%\item diagram methods

Purely data-driven methods are commonly known as feature selection or dimension
reduction. The goal is to find a subset of dimensions that are critical to
understanding the dataset. \ttwnote{issues of understanding the mapping}
\ttwnote{topology throws away the parameters} Projections also synthesize the
dataset into new dimensions to show using visual channels. \ttwnote{example}
\ttwnote{also figure out where to put projection}

View-based methods try and produce multiple linked views of a multi-dimensional
dataset from different angles. Each view shows a subset of the dimensions.
This way we can use a proper set of visual channels for each view.  We use
interaction to link these different views. These multiple, coordinated, linked
views have been one of the biggest success stories from the visualization
community~\cite{ref}. The traditional HyperSlice~\cite{Wijk:1993} technique
falls into this category. Each panel of the HyperSlice view shows two of the
input dimensions and the value is encoded with color.

\ttwnote{conclude and transition}

