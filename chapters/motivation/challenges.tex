
\section{Challenges with multi-dimensional spaces}
\label{sec:multi-d-challenges}

\ttwnote{this section should be about why multi-D is hard}

\ttwnote{too unorganized... maybe turn it into a technique overview section}

Given the importantance of multi-dimensional data analysis, a number of
projects from the visualization community, as well as others, have worked on
this problem. 
\ttwnote{rewrite this paragraph}
\ttwnote{introduce each section of the motivation with what's to come} 

Understanding multi-dimensional space is difficult. As humans, we simply do not
have the spatial analogs in more than three dimensions. A number of methods
have been developed to extract specific features from the multi-dimensional
object. For example, when studying polytopes, the number of faces and
symmetries is very important~\cite{Ziegler:2012}. However, these only produce
an answer without sufficient context. They do not necessarily give any
intuition as to how to transfer our three-dimensional knowledge to
multi-dimensional spaces.

\ttwnote{how visualization can help}
\ttwnote{some sort of insight generation from context?}

The difficulty of visualizing a continuous multi-dimensional space on a
two-dimensional screen brings a number of challenges. We treat each dimension
separately, thus, we need several different visual channels. The ranking of
effectiveness of visual channels (shown in \autoref{tbl:visual_encodings}) was
proposed by Bertin~\cite{Bertin:1967} and confirmed through experiments by
Cleveland and McGill~\cite{Cleveland:1984}, Mackinlay~\cite{Mackinlay:1986},
and Heer and Bostock~\cite{Heer:2010}.  Munzner~\cite{Munzner:2014} provides a
summary of the results. In addition, there are many others. We are also limited
in how many channels we can use simulataneously. According to
Ware~\cite{Ware:2004}, many channels, such as red and green are not visually
seperable. Furthermore, each dimension of the multi-dimensional object under
study is often treated equally. For example, no particular axis of a polytope
is more important than any other.  We should encode each dimension using
equally weighted effectiveness channels.  With fewer channels available than
data dimensions we either need to reduce the data or use multiple views to
properly visualize the data.

\begin{table}
  \caption{Rankings of visual encodings of quantiative data}
  \label{tbl:visual_encodings}
  \begin{tabular}{llll}
    \cite{Bertin:1967} & \cite{Cleveland:1984} & \cite{Mackinlay:1986} & \cite{Munzner:2014} \\
     Position & Position along a common scale & Position & Position on common scale \\
     Size & Position along identical, nonaligned scales & Length & Position on unaligned scale \\
     (Grey) Value & Length & Angle & Length (1D size) \\
     Texture & Direction & Slope & Tilt/angle \\
     Color & Angle & Area & Area (2D size) \\
     Orientation & Area & Volume & Depth (3D position) \\
     Shape & Volume & Density & Color luminance\\
     & Curvature & Color saturation & Color saturation \\
     & Densities & Color hue & Curvature \\
     & Shading & & Volume (3D size) \\
     & Color saturation &           & 
  \end{tabular}
\end{table}

Purely data-driven methods are commonly known as feature selection or dimension
reduction. The goal is to find a subset of dimensions that are critical to
understanding the dataset. Topological techniques take this a step further.
They discard all spatial information about the dataset and only concentrate on
the difference in function value, as in the Morse-Smale
complex~\cite{Gyulassy:2012a}, or evolution of contours, as in the contour
tree~\cite{Carr:2003a}.  Projections also synthesize the dataset into new
dimensions to show using visual channels. Principal component
analysis~\cite{PCA} is a popular choice in this area. This rotates the space
and thus produces new set of dimensions that are a linear combination of the
input dimensions. Even this relatively simple operation (a rotation) can be
difficult to understand. For example, iPCA~\cite{Jeong:2009a} was a tool to
help users understand the effects of the dimensional transformation.

View-based methods try and produce multiple linked views of a multi-dimensional
dataset from different angles. Each view shows a subset of the dimensions.
This way we can use a proper set of visual channels for each view.  We use
interaction to link these different views. These multiple, coordinated, linked
views have been one of the biggest success stories from the visualization
community~\cite{Rao:1994}. The traditional HyperSlice~\cite{Wijk:1993}
technique falls into this category. Each panel of the HyperSlice view shows two
of the input dimensions and the value is encoded with color.

\ttwnote{conclude and transition}

