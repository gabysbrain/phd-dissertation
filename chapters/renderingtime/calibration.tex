\section{Fitting}
\label{sec:fitting}

With a proper mathematical formulation of the scene geometry in terms of how
many pixels are produced on screen, I now describe the second missing piece of
the prediction, namely, a procedure for tuning this formula to a particular
user's implementation and hardware configuration.  The values $t_f$ and
$t_\text{H}$ in \autoref{eq:exptotal} are dependent on a particular hardware.
Hence, I engage in an empirical stage to determine these values for a
particular rendering environment.  For the purposes of these timings I set my
correlation cutoff value (see \autoref{sec:scene_geometry}), $\epsilon$, to $1
\times 10^{-9}$. This value allows me to link the kernel radius, $r$, with the
kernel bandwidth parameter in the GP model.

The architecture of each GPU is different and what is efficient on one 
architecture may not carry over to another.
Therefore, it is
impossible to argue from first-principles how to derive $t_f$ and $t_\text{H}$ 
for a
particular GPU. I fit these parameters by doing a regression analysis
on empirical results obtained from examining the time to render a frame for
various values of $d$, $N$, and $r$. While the specific values I derive are 
specific to a particular architecture,
the method we present is applicable
elsewhere. 

In order to fit \autoref{eq:exptotal} I note that the first term
represents the number of points we need to check to be filtered. This is 
constant
with respect to the kernel size, $r$. The second term, the drawing
time, increases with respect to $r$. The filtering time dominates
for small values of $r$ while the drawing time dominates for larger values of 
$r$.
Therefore, there will be a point in terms of number of fragments drawn, 
designated $a$, at which point the dominant term will change from the first to
the second. In order to fit this behavior I used a segmented regression
model which changes behavior according to the value of a \{0,1\} indicator
function, $I(\text{frags}<a)$, where $\text{frags}$ is the total number of 
fragments drawn. This 
function returns 0 if $\text{frags} < a$. I form the regression
formula as:
\begin{align}
  t_\text{render} &= N {d \choose 2} t_f + I(\text{frags} > a) t_\text{H} \text{frags}  \text{.}
  \label{eq:calib-acttotal-H}
\end{align}

This formula contains three parameters to be estimated: $t_f$ (the time to filter one sample
point), $t_\text{H}$ (the time to render one fragment), and $a$ (the crossover 
point).

\subsection{Sampling}
\label{sampling}

For each dimension, I must ensure that I have good coverage of the number of
fragments being drawn.
Hence, I must choose different values of $r$ for each $d$.
In order to 
obtain
a sensible range of values for $\expfrags$ we begin by using dimension 3
as a baseline and vary radii from $0$ to $0.5$ in that dimension to come 
up with a reasonable
range of $\expfrags$.
Given this desired range of fragments,
for the remaining dimensions I numerically invert
\autoref{eq:expfrags},
given $d$ and hence, obtain a range of radii $r$. 

The final issue is that for each setting of $d$, $N$, and $r$ I must
generate enough iterations such that the average number of points affecting
the slices, $N'$, converges to the theoretical expected value, $E[N']$. 
This is the expected number of points need to be drawn over all possible
uniform distributions of points and viewpoints. In my case we found that
20 viewpoint changes, redrawing the screen for a fixed viewpoint 5 times,
and 3 sample
point distributions for a total of 300 timing measures resulted in good
convergence.

I compute the number of fragments drawn on screen internally by the 
application.  I can do this because we know the position of the focus
point, locations of all the sample points, and the kernel information.  
The number of fragments in the calculations is the percentage the quad takes
up on the subplot.
In other words, if a quad takes up an entire subplot then
it has area $1$.
This measurement serves as a proxy 
for the number of fragments generated in the GPU.  
OpenGL offers a query 
object, GL\_SAMPLES\_PASSED, that should return the number of fragments 
needed to draw
the screen.  This is very convenient and would correctly account for the 
rasterization method used.  However, in my experiments this query would 
return several different values for the exact same scene, which does not
make sense.
Due to this inconsistency, I used an internal calculation.

I can record the rendering time with either the CPU timer or the GPU timer.
The CPU timer better represents the user's perception of how long it takes
to draw the screen since it includes the time necessary to transfer data back
and forth to the GPU.  However, this timer is much noisier than the GPU
timer.  In my tests I found that on average the CPU timing differed from
the GPU timing by a constant amount.  Therefore, I used the GPU timer for
the timing.  The GPU timer is still quite noisy however and in order to 
smooth out this noise I redrew each screen five times for each change of
viewpoint and then averaged the times.

\subsection{Final model}
\label{finalmodel}

If we then apply these estimates of filter and draw time to 
\autoref{eq:exptotal}, the full form of the prediction 
model, conditional on the dimension, $d$, is
\begin{align*}
  t_{total}(d, N, r) &= 
      t_f(d) {d \choose 2} N + I(\text{frags}>a) t_\text{H}(d) E[N']E[q] \\
   &= t_f(d) {d \choose 2} N + I(\text{frags}>a) t_\text{H}(d) {d \choose 2} N \exppts{r} \expfrags \\
      t_{total}(d, N, r) &= {d \choose 2} N \\
                         & \left[t_f(d) + I(\exppts{r} \expfrags > a(d)) t_\text{H}(d) \exppts{r} \expfrags \right] \text{,}
    \numberthis \label{eq:expanded-regression}
\end{align*}
where $t_f(d)$, $t_\text{H}(d)$, and $a(d)$ are the parameters fit
and $\exppts{r}$ and $\expfrags$ are from \autoref{eq:exppoints} and 
\autoref{eq:expfrags} respectively.
The parameters are conditional on the dimension, $d$, because I 
fit each dimension separately so there is a different value of $t_f$, 
$t_\text{H}$, and $a$ for each dimension.
I fit these parameters using segmented regression as described in
\autoref{sec:datafitting}.

