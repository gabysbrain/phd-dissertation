

For this derivation, assume a $d$-dimensional centred unit
cube\footnote{Note, that in \autoref{chp:rendering} I specified a
non-centred unit cube $[0, 1]^d$. However, the centering does not impact the
result, but is mathematically easier to deal with.} $[-0.5, 0.5]^d$ with
coordinate axes $x_1, x_2, \ldots, x_d$. Without loss of generality, we
specify a 2D slice by a $(d-2)$-dimensional (focus) point $\vs$ and 
assume that the two additional coordinates are the coordinates within the
slice.

As explained in \autoref{sec:expectedtotaltime}, $\exppts{r}$ is the expected
percentage of points within distance $r$ from a single 2D slice in $d$
dimensions. In the case of a uniform point distribution, this essentially
measures the volume of a slab with thickness $2r$ around the slice. The extent
of this slab in the $d$th and $d-1$ dimensions is one, since we have a
$d$-dimensional unit cube. Hence, the volume of the slab is the
$(d-2)$-dimensional volume $V(\vs)$ around the $(d-2)$-dimensional point $\vs$
multiplied by one for each direction $(d-1)$ and $d$:
\begin{equation}
  V(\vs) = 1 \cdot 1 \cdot \int_{[-0.5,0.5]^{d-2}} B_r(\vs-\vx) d\vx \text{,}
   \label{eq:vs}
\end{equation}
where, $B_r$ is the constant $(d-2)$-ball with radius $r$:
\begin{align*}
  B_r(\vx) &= \left\{ 
  \begin{array}{l l}
    1 & \quad \text{if $||\vx||<r$}\\
    0 & \quad \text{otherwise} \text{.}\\
  \end{array} \right. 
\end{align*}

Considering the $(d-2)$-dimensional centred unit box $\Pi(\vx)$, we express \autoref{eq:vs} as a convolution:
\begin{align*}
  V(\vs) &= \int_{[-0.5,0.5]^{d-2}} B_r(\vs-\vx) d\vx\\
          &= \int_{[-\infty,\infty]^{d-2}} \Pi(\vx) B_r(\vs-\vx) d\vx \\
          &= \Pi * B_r(\vs) \text{.}
\end{align*}

Since $\exppts{r}$ is the average volume over all possible slice positions within the $(d-2)$-dimensional unit cube, we conclude:
\begin{align*}
  \exppts{r} &= \int_{[-0.5,0.5]^{d-2}} V(\vs) d\vs \\
  	&= \int_{[-\infty,\infty]^{d-2}} \Pi(\vec{0}-\vs) \Pi * B_r(\vs) d\vs \\
  \exppts{r} &= \left( \Pi * \Pi * B_r \right) \left(\vec{0}\right) \text{.}
\end{align*}

The beauty of this last expression is, that it is a convolution of three different piecewise-constant (the constant being one) functions evaluated at zero. This allows us to reinterpret this expression as an integral of the convolution of two of these functions over the domain of the third function:
\begin{equation}
  \exppts{r} = \int_{B_r} T(\vx) d\vx \text{,}
  \label{eq:Erd}
\end{equation}
where $T(\vx) = \Pi * \Pi (\vx)$ is the convolution of two unit-cubes or the $(d-2)$-dimensional triangle function\footnote{Also known as the tensor-product of the linear B-spline or the $(d-2)$-linear interpolator}. Hence, in the positive orthant, we can write:
\begin{align*}
  T^{+}(\vx) &= \prod_{i=1}^{d-2} \max(0,(1-x_i)) \text{,}
\end{align*}
where $\vx = (x_1, x_2, ..., x_n)$ and, in general, $f^{+}(\vx)$ shall denote the positive orthant of some function $f(\vx)$.


In the remainder of this appendix we will evaluate the integral in \autoref{eq:Erd} for the case of HyperSlice. Since this is an integral in a $(d-2)$-dimensional space, we will, for brevity, be using $n$ to equal $(d-2)$ in the remainder of this appendix.
In the case of the HyperSlice the distance metric is the $L^2$ norm and the volume $B_r$ is an $n$-dimensional sphere of radius $r$. Hence, \autoref{eq:Erd} becomes:
\begin{equation}
  \exppts{r} = 2^n \int_{B^{+}_r} T^{+}(\vx) d\vx \text{.}
  \label{eq:ErdConvH}
\end{equation}
We will solve this integral using polar coordinates.

\section{Polar coordinates in $n$ dimensions}

Expressing $\vx = (x_1, x_2, ..., x_n)$ in polar coordinates $\vvarphi = (R, \phi_1, \phi_2, \dotsc, \phi_{n-1})$ can be done as follows:
\begin{align*}
x_1 &= R \cos(\phi_1)\\
x_2 &= R \sin(\phi_1) \cos(\phi_2)\\
x_3 &= R \sin(\phi_1) \sin(\phi_2) \cos(\phi_3)\\
\vdots\\
x_{n-1} &= R \sin(\phi_1) \cdots \sin(\phi_{n-2}) \cos(\phi_{n-1})\\
x_n &= R \sin(\phi_1) \cdots \sin(\phi_{n-2}) \sin(\phi_{n-1}) \text{,}
\end{align*}
where $\phi_i \in [0, \pi]$ for $i = 1, \dotsc, n-2$ and $\phi_{n-1} \in [0, 2\pi]$.

The Jacobian, needed to properly substitute the integration variables in \autoref{eq:ErdConvH} of the transformation from spatial coordinates $\vx$ to polar coordinates $\vvarphi$, can be computed as follows:
\begin{align*}
  \frac{d\vx}{d\vvarphi} &= R^{n-1} \prod_{i=1}^{n-1}\sin^{n-1-i}(\phi_i) \\
  d\vx &= R^{n-1} dR \prod_{i=1}^{n-1}\sin^{n-1-i}(\phi_i) d\phi_i
  \text{.}
\numberthis \label{eq:spherical}
\end{align*}

%Note, that the product is going all the way to $(n-1)$ and it still works fine!

\section{Derivation}

In this section, we will assume that the radius $r$ never grows above one. This is a reasonable assumption, which holds for the experiments I performed. We can now simplify \autoref{eq:ErdConvH} as follows:
\begin{align*}
\exppts{r} &= 2^n \int_{B^{+}_r} \prod_{i=1}^{n} \max(0,(1-x_i)) d\vx \\
  &= 2^n\int_{B^{+}_r} \prod_{i=1}^{n} (1-x_i) d\vx \\
  &= 2^n \int_{B^{+}_r} \sum_{i=0}^{n} (-1)^{i} t_i(\vx) d\vx \\
  &= 2^n \sum_{i=0}^{n} (-1)^{i} \int_{B^{+}_r}  t_i(\vx) d\vx
     \text{,}
\numberthis \label{eq:ti}
\end{align*}
where $t_i(\vx)$ is the sum of all products of $i$ distinct coordinates. 
In other words,
\begin{align*}
t_0(\vx) &= 1 \\
t_1(\vx) &= \sum_{j=1}^n x_j \\
t_2(\vx) &= \sum_{j,k=1; j\ne k}^n x_j x_k \\
\vdots
\end{align*}

Because of symmetry in the first orthant around any axis $x_i = x_j$ of the hypersphere, we have:
\begin{align*}
\int_{B^{+}_r}  x_j x_k d\vx &= \int_{B^{+}_r}  x_1 x_k d\vx = \int_{B^{+}_r}  x_1 x_2 d\vx \text{,}
\end{align*}
which holds for any product of arbitrary terms. Hence, we can simplify \autoref{eq:ti} in the following way:
\begin{align*}
\exppts{r} &= 2^n \sum_{i=0}^{n} (-1)^{i} \int_{B^{+}_r}  t_i(\vx) d\vx \\
  &= 2^n \sum_{i=1}^{n} (-1)^{i} {n \choose i} \int_{B^{+}_r} \prod_{k=1}^i x_k  d\vx + 2^n  \int_{B^{+}_r}  d\vx \\
  &= 2^n \sum_{i=1}^{n} (-1)^{i} {n \choose i} A_i + B \text{.}
\end{align*}

$B$ is the volume of an $n$-dimensional sphere of radius $r$:
\begin{align*}
B &= 2^n  \int_{B^{+}_r}  d\vx =  \int_{B_r}  d\vx = \frac{\pi^\frac{n}{2} r^n}{\Gamma(\frac{n}{2}+1)} \text{.}
\end{align*}

We are left to compute the product integrals $A_i$, for which we use polar coordinates. For $i \le n$, we have (using \autoref{eq:spherical}):
\begin{align*}
A_i &= \int_{B^{+}_r} \prod_{k=1}^i x_k  d\vx \\
  &= \int_{\vvarphi^{+}} \prod_{k=1}^i x_k(\vvarphi)  R^{n-1} dR \prod_{j=1}^{n-1}\sin^{n-1-j}(\phi_j) d\phi_j \\ 
  &= \int_{\vvarphi^{+}}  R^i \prod_{k=1}^i \cos(\phi_k)\sin^{i-k}(\phi_k)  R^{n-1} dR \prod_{j=1}^{n-1}\sin^{n-1-j}(\phi_j) d\phi_j \\ 
  &=  \int_{\vvarphi^{+}} R^{n+i-1} dR \prod_{k=1}^i \cos(\phi_k)\sin^{i-k}(\phi_k) \prod_{j=1}^{n-1}\sin^{n-1-j}(\phi_j) d\phi_j \\
  &=  \int_{\vvarphi^{+}} R^{n+i-1} dR \prod_{k=1}^i \cos(\phi_k)\sin^{n-1+i-2k}(\phi_k) \prod_{j=i+1}^{n-1}\sin^{n-1-j}(\phi_j) d\phi_j \quad\quad \\
  &=  A^R_i \prod_{k=1}^i A^c_{i,k}  \prod_{j=i+1}^{n-1} A^s_{i,j}
      \text{,}
  \numberthis \label{eq:threeAs}
\end{align*}
where
\begin{align*}
  A^R_i &=  \int_0^r R^{n+i-1} dR \\
  A^c_{i,k} &= \int_0^{\pi/2} \cos(\phi_k)\sin^{n-1+i-2k}(\phi_k)  \\
  A^s_{i,j} &= \int_0^{\pi/2} \sin^{n-1-j}(\phi_j) d\phi_j
      \text{,}
\end{align*}
where we used the fact that the proper positive orthant integration bounds by $\vvarphi^{+} = [0,R]\times[0,\pi/2]^{n-1}$.

Solving for these three types of integrals yields:
\begin{align*}
A^R_i &= \int_0^r R^{n+i-1} dR = \frac{1}{n+i} r^{n+i} \text{,}
\end{align*}
\begin{align*}
  A^c_{i,k} &= \int_0^{\pi/2} \cos(\phi_k)\sin^{n-1+i-2k}(\phi_k) \\
  	&= \frac{1}{n+i-2k} \left. \sin^{n+i-2k}(\phi_k)\right|_0^{\pi/2} \\
	&= \frac{1}{n+i-2k}  \text{,}
\end{align*}
as well as
\begin{align*}
  A^s_{i,j} &= \int_0^{\pi/2} \sin^{n-1-j}(\phi_j) d\phi_j \\
  	&= \left. -\cos(\phi_j)F_1(0.5,(j-n+2)/2, 1.5, \cos^2(\phi_j))\right|_0^{\pi/2} \\
	&= F_1(0.5,(j-n+2)/2, 1.5, 1) \\
	&= \frac{\Gamma(3/2)\Gamma((n-j)/2)}{\Gamma(1)\Gamma(0.5+(n-j)/2)}\\
	&= \frac{\pi^{1/2}}{2} \frac{\Gamma((n-j)/2)}{\Gamma((n-j+1)/2)}
       \text{,}
\end{align*}
where $F_1$ is the hypergeometric function and $\Gamma$ is the gamma function.
Putting this all together (for $i \le n$) into \autoref{eq:threeAs}, we get:
\begin{align*}
  A_i &= A^R_i \prod_{k=1}^i A^c_{i,k}  \prod_{j=i+1}^{n-1} A^s_{i,j} \\
  	&=  \frac{1}{n+i} r^{n+i} \prod_{k=1}^i \frac{1}{n+i-2k}  \prod_{j=i+1}^{n-1} \frac{\pi^{1/2}}{2}\frac{\Gamma((n-j)/2)}{\Gamma((n-j+1)/2)} \\
	&=  r^{n+i} \prod_{k=0}^i \frac{1}{n+i-2k}  \frac{\pi^{\frac{n-1-i}{2}}}{2^{n-1-i}}\frac{\Gamma(1/2)}{\Gamma((n-i)/2)} \\
	&=  r^{n+i} \frac{\pi^\frac{n-1-i}{2}}{2^{n-1-i}}\frac{\pi^{1/2}}{\Gamma((n-i)/2)} \prod_{k=0}^i \frac{1}{n+i-2k} \\
	&=   \frac{\pi^\frac{n-i}{2}}{2^{n-1-i}}\frac{r^{n+i}}{\Gamma((n-i)/2)} \prod_{k=0}^i \frac{1}{n+i-2k}
    \text{.}
    \numberthis \label{eq:PreAi}
\end{align*}

This formula can be further simplified by noting that if we expand out 
the product we get,
\begin{align*}
  \prod_{k=0}^i \frac{1}{n+i-2k} 
    &= \frac{1}{(n+i)(n+i-2)(n+i-4)\cdots(n+i-2i)} \\
    &= \frac{1}{\frac{2^{i+1}}{2^{i+1}}
                (n+i)(n+i-2)(n+i-4)\cdots(n+i-2i)} \\
    &= \frac{1}{2^{i+1}
                \left(\frac{n+i}{2}\right)
                \left(\frac{n+i}{2} - 1\right)
                \left(\frac{n+i}{2} - 2\right)
                \cdots
                \left(\frac{n+i}{2} - i\right)} \\
    &= \frac{1}{2^{i+1}
                \left(\frac{n+i}{2}\right)
                \left(\frac{n+i}{2} - 1\right)
                \left(\frac{n+i}{2} - 2\right)
                \cdots
                \left(\frac{n-i}{2}\right)} \text{.}
\end{align*}
This expansion combined with $\frac{1}{\Gamma((n-i)/2)}$ means that instead
of,
\begin{align*}
  \frac{1}{\Gamma((n-i)/2)} \prod_{k=0}^i \frac{1}{n+i-2k}
  \text{,}
\end{align*}
we can write,
\begin{align*}
  \frac{1}{2^{i+1}\Gamma((n+i)/2)}
  \text{.}
\end{align*}
Therefore, we can simplify \autoref{eq:PreAi} as,
\begin{align*}
  A_i &= \frac{\pi^\frac{n-i}{2}}{2^{n-1-i}}\frac{r^{n+i}}{\Gamma((n-i)/2)} \prod_{k=0}^i \frac{1}{n+i-2k} \\
      &= \frac{\pi^\frac{n-i}{2}}{2^{n-1-i}}\frac{r^{n+i}}{2^{i+1}\Gamma(\frac{n+i}{2} + 1)} \\
      &= \frac{\pi^\frac{n-i}{2}}{2^{n}}\frac{r^{n+i}}{\Gamma(\frac{n+i}{2} + 1)}
           \text{.}
           \numberthis \label{eq:Ai}
\end{align*}

Intuitively, $A_i$ measures the expected area of a hypersphere that is clipped
by the edges of the parameter space for all possible combinations of the
$i$-dimensional subspaces.

Finally, we can put everything together:
\begin{align*}
\exppts{r} &= 2^n \sum_{i=1}^{n} (-1)^{i} {n \choose i} A_i + B \\
  &= 2^n \sum_{i=1}^{n} (-1)^{i} {n \choose i} A_i + B \\
  &= 2^n \sum_{i=1}^{n} (-1)^{i} {n \choose i}\left[ 
       \frac{\pi^\frac{n-i}{2}}{2^{n}}\frac{r^{n+i}}{\Gamma(\frac{n+i}{2} + 1)}
     \right] + B \\
  &= \sum_{i=1}^{n} (-1)^{i} {n \choose i}\left[ 
       \frac{\pi^\frac{n-i}{2}r^{n+i}}{\Gamma(\frac{n+i}{2} + 1)} 
     \right] 
     + \frac{\pi^\frac{n}{2} r^n}{\Gamma(\frac{n}{2}+1)}
        \text{.}
  \numberthis \label{eq:pre_exppts}
\end{align*}
Now we note that if $i=0$,
\begin{align*}
  (-1)^{i} {n \choose i}\left[ 
       \frac{\pi^\frac{n-i}{2}r^{n+i}}{\Gamma(\frac{n+i}{2} + 1)} 
     \right] 
    &= (-1)^{0} {n \choose 0}\left[ 
       \frac{\pi^\frac{n-0}{2}r^{n+0}}{\Gamma(\frac{n+0}{2} + 1)} 
     \right] \\
    &= \frac{\pi^\frac{n}{2}r^{n}}{\Gamma(\frac{n}{2} + 1)} 
       \text{.}
      \numberthis \label{eq:exppts_0}
\end{align*}
Which is the $n$-dimensional volume of the ball.
If we include $i=0$ in the summation and substitute \autoref{eq:exppts_0} 
into \autoref{eq:pre_exppts} then we can write $\exppts{r}$ as,
\begin{align*}
\exppts{r} 
  &= \sum_{i=0}^{n} (-1)^{i} {n \choose i}
         \frac{\pi^\frac{n-i}{2}r^{n+i}}{\Gamma(\frac{n+i}{2} + 1)} 
            \text{.}
  \numberthis \label{eq:exppts}
\end{align*}

