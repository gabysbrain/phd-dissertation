\section{Limitations and future work}
\label{sec:conclusion}

In this paper we have presented the characteristics of data used for
analyzing computer simulations under the 
\emph{design and analysis of computer experiments} 
framework~\cite{Santner:2003}.  We investigate
how interactive rendering times may be used in this framework using the
HyperSlice~\cite{Wijk:1993} rendering technique implemented on the GPU.
We then describe a method using both the scene geometry and estimates of the
user's machine capabilities in order to make an accurate prediction of 
the rendering time.
We find that timing, especially using wall-clock time, is extremely noisy
and makes fitting very difficult.  It is much more reliable to use the
timer on the GPU itself, however one must take into account the time needed
to return back from the GPU which in our experiments is about 1ms.

In the future, we will investigate how to reduce the number of trials
needed to properly fit the formula.  Currently each dimension
takes about 6 hours to complete and requires the machine to be dedicated
to running the timing code.  Reducing the number of trials will help to
alleviate this time consuming task.

It would also be interesting to extend our approach to the analysis of
HyperSlice rendering in general. The basic geometrical operation in our
mathematical model is slicing multi-dimensional spheres with 2D planes and
estimating the area. Therefore our method should work directly with any radial
basis function reconstruction technique like the work by Hong et
al.~\cite{Hong:2006} although we have not directly tested this.  We would also
like to extend our mathematical model to take the shape of the reconstruction
primitive into account. This would allow us to analyze the timing of HyperSlice
rendering using a much more broad set of reconstruction methods like 
nearest neighbor or linear regression. 
The framework can also be used to estimate the rendering time of density
estimation by setting $f(x_i) = 1$ in \autoref{eq:kernel_regression}. Repeating the
analysis using box primitives would allow us to estimate the time complexity
of some of the recent real-time large-data aggregation and visualization 
methods like imMens~\cite{Liu:2013} and NanoCubes~\cite{Lins:2013}. Both these
methods use rectangular binning in their density estimation.

We will also implement our subsampling strategy.
There is a lot of overdrawing occurring which does not contribute at
all to the final plot as the color channel is effectively maxed out, 
especially as the value of $N$ and $r$ increases.  By detecting when this 
occurs and only rendering the first few points we could improve rendering
efficiency.

