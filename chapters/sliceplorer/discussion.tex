
\section{Discussion}

The above examples illustrated that the technique of 1D slices as presented are
quite flexible and useful for various low- and high-level tasks.  However, I do
not intend to claim that it is the only and best method for all problems out
there. Rather, I would like to argue that it is a valuable (and thus far
overlooked) technique in a toolbox of visual inspection methods for
multi-dimensional functions. I hope that this work inspires a discussion and
exploration of guidelines for tasks, proper visual encoding, and interaction
techniques for various application areas. Along these lines I would like to put
forth our current experience with various techniques.

\textbf{Topological techniques are helpful for a global overview}: Topological
techniques allow us to compare \emph{between} optima but are not as good at
evaluating the area \emph{around} an optimum since these areas are typically
abstracted away.  Topological spines attempts to compensate for this by showing
the area covered by a particular optimum as an area around the node. However,
many of the tasks like ``correlate'' and ``cluster'' are best served by viewing
the response manifold directly. In a larger system, the topological techniques
could be used effectively as a global overview of the function with a
HyperSlice or 1D slice showing local context. Selecting a point in the
topological view would change the focus point in the local view.

\textbf{HyperSlice is good when you need to show 2D interactions}: HyperSlice
is the only technique that can display more than one dimension of data
interaction.  So, if this is a requirement then HyperSlice is the best option.
However, one can use 1D slices to get a general overview of the dependence of
the function on each dimension. The dimensions that are not interesting
because, for example, the function is not sensitive to them could easily be
eliminated from further consideration. This would reduce the number of subplots
that we need to view in the HyperSlice plot.

\textbf{1D slices should be used for a ``first pass'' visualization}: 1D slices
addresses many of the tasks that a user wants to perform. The technique does a
very good job on a wide variety of tasks. 1D slices are easy to
implement, easy to understand, and the static view provides a lot of
information.

