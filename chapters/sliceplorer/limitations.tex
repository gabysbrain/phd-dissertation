\section{Limitations and future work}
%\label{sec:limitations}

The 1D slice view consists of a projection of many lines.  the distribution of
slices are shown through direct projection. Techniques like contour
boxplots~\cite{Whitaker:2013} and curve boxplots~\cite{Mirzargar:2014} build a
distribution model of curves which could help to address the ``characterize
distribution'' task in \autoref{tbl:task_list}. However, neither of these or
any of the other time curve visualization techniques have been applied to
multi-dimensional functions. Evaluating these techniques for this purpose is an
exciting topic for future work.

When developing the 1D continuous slicing technique I only considered
multi-dimensional continuous scalar functions in terms of requirements, 
tasks, and comparisons. I do not consider multi-field 
(i.e.\ functions with multiple outputs) or complex-valued functions in the
analysis. There are multi-field topology techniques to address 
this~\cite{Duke:2012,Huettenberger:2014,Carr:2015} which I do not consider
but the technique and analysis would need to be extended to this domain.
This is left for future work.

The x-axis of each 1D slice is independent of the x-axes of the other 1D
slices. This allows each plot to scale individually if the range of inputs have
different values.  The x-axis and y-axis automatically change to incorporate
their respective minimum and maximum ranges. While the x-axis scales itself
independently, the y-axis is the same for each plot.  This is also the default
behavior in many of-the-shelf plotting packages. The plots will adjust
automatically to shifts.  For the x-axis we use axis-aligned projections.
Therefore, the views are sensitive to rotational transformations of the
function. 

Finally, the Sliceplorer technique is also based on sampling, just like the 
techniques used in the comparison.
As with any technique based on sampling one
must be careful to take an adequate number of samples in order to properly
capture all desired behavior.
%\ttwnote{add more about sampling limitations?}
If the function is not smooth we may see a slice that is an ``outlier,'' i.e.\
one slice is much higher or lower than all the others. In this case all other
slices will be compressed into either the top or bottom of the chart. This is
often a problem with many common visualization techniques like bar graphs or
scatterplots and can be addressed with log scaled axes, for example.

