\section{Related Work}\label{sec:background}

The question of how to comprehend multi-dimensional data is a heavily
researched area in visualization. There are two principal approaches,
projection techniques (such as scatterplots) and slicing techniques (such as
HyperSlice). Projection techniques generally show all of the data and,
therefore, represent a more global view. On the other hand, slicing techniques
present more of a local view around a point of interest (which we call focus
point). The lion's share of previous work is concerned with projection
approaches of discrete data. In contrast, I focus on continuous
multi-dimensional scalar functions and seek to combine the strengths of
projection and slicing. Given the focus on continuous data, the options for a
visual exploration are limited and could be categorized into three areas (a)
discretization, (b) local methods, (c) global methods.

\subsection{Discretization}

There are a number of different approaches to display discretized
multi-dimensional functions. The typical approaches are scatterplot
matrices~\cite{Hartigan:1975}, parallel coordinates~\cite{Inselberg:1985}, star
coordinates~\cite{Kandogan:2000}, and RadVis~\cite{Hoffman:1997}. 
Star coordinates and RadVis were generalized into one framework by Lehmann
and Theisel~\cite{Lehmann:2016a}.
These can all be combined with a variety of dimensionality
reduction techniques~\cite{Holbrey:2006}. However, all of these seem
inappropriate if the mental model of the function we are studying is a continuous
one. In such a case all of these projection techniques would fail to properly
convey the complexity of the underlying continuous phenomenon. Hence, while
discretization seems like an easy way out, it is not a proper alternative for
studying \emph{continuous} multi-dimensional functions, such as regression
functions or classification boundaries. Here, one of the main concerns is
understanding a continuous phenomenon and a good visualization design should
thus respect this mental model~\cite{Tory:2004a,Sedlmair:2012,Liu:2010a}.

\subsection{Local methods}

The idea of a local technique is to focus on a part of the function.
Interaction is used to explore other parts of the function. One of the oldest
approaches here is the HyperSlice technique by van Wijk and van
Liere~\cite{Wijk:1993}.  HyperSlice is a technique where the function is shown
directly but in multiple 2D slices laid out similar to a scatterplot matrix. In
many ways, Sliceplorer was inspired by this work. One of the drawbacks of
HyperSlice is that one has to choose a focus point --- a point common to all 2D
slices. Exploring the full data set then shifts over to exploring all possible
focus points. Although not created for HyperSlice specifically,  techniques
like the grand tour~\cite{Asimov:1985}, projection pursuit~\cite{Huber:1985},
and optimal sets of projections~\cite{Lehmann:2015b} might be appropriate to
tackle this issue.
All of these approaches are
still local though. A mental image of the global function can only be built up
over time and with mental effort by browsing through the focus points. Our
approach seeks to overcome this limitation while keeping the benefits of ease
of understanding.  

Note that for some tasks a local view might in fact suffice, such as when one is interested in the robustness of an extrema value. 
For example, Tuner~\cite{Torsney-Weir:2011} used 2D
continuous slices, letting the user navigate them via selecting Pareto optimal focus points in a separate view. 
Berger et al.~\cite{Berger:2011} use coordinated views of
scatterplots and parallel coordinates to show additional (continuous) prediction uncertainty. Having a discrete approach provided extra space to display information about the prediction uncertainty for the currently selected point.

\subsection{Global methods}

The visualization community has developed many global views of
multi-dimensional continuous functions. Continuous
scatterplots~\cite{Bachthaler:2008} and continuous parallel
coordinates~\cite{Heinrich:2009} can encode a multi-dimensional density field
into either two dimensions or more than two dimensions respectively.  Here, we
are not as concerned with the density field, rather we are concerned with the
manifold created from a continuous multi-dimensional scalar function. 

Topological methods like topological
spines~\cite{Correa:2011}, the work by Gerber et al.~\cite{Gerber:2010}, and
contour trees~\cite{Carr:2003a} extract extrema and saddle points from a
function and then show these. These methods are good for seeing the relative
relation of extrema in a function. However, they do not work for important
tasks like robust optimization. Here, one does not necessarily want to find the
global optimum but wants an optimum in a relatively ``flat'' area of the
parameter space. 

With Sliceplorer, we use line plots together with the widely-used technique of
projection to overdraw 1D slices. This approach is similar to the work by Hall
et al.~\cite{Hall:2014} but differs in two major ways. (1) They showed 2
primary dimensions with slices and then used the third for color limiting their
view to three dimensions and (2) they were concerned with isosurface
extraction. This technique can scale to any number of dimensions and the
evaluation is based on a much broader set of tasks and applications, such as
parameter space analysis. 

