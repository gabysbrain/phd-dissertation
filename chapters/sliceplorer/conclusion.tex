\section{Conclusion}

In this paper we have presented Sliceplorer, a visualization method for
multi-dimensional functions based on one-dimensional slices. We defined a task
taxonomy specific to multi-dimensional continuous functions and found that,
while some state of the art techniques are very good at  addressing specific
tasks, our method supports a wide variety of tasks. Consequently, our technique
may be a good first pass when visualizing multi-dimensional continuous functions. 
It is easy to implement, easy to understand, and addresses a greater variety
of tasks than any other technique. 

\section*{Acknowledgments}
We wish to thank the members of the VDA lab for their helpful feedback and
support with this project especially Peter Ruch, Michael Oppermann, and
Patrick Wolf. We also wish to thank Harald Piringer, Eduard Gr{\"o}ller, Ivan
Viola for their feedback on the project.  The authors wish to especially
thank Johanna Schlereth for her invaluable help with making the video.
This work was partly supported by the Vienna Business Agency.

